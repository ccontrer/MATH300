\documentclass[12pt,oneside]{book}


\usepackage{remotelecture}
\usepackage{amsmath}
\usepackage{amsfonts}
\usepackage{amssymb}
\usepackage{textcomp}
\usepackage[margin=60pt]{geometry}
\usepackage{fancyhdr}
\usepackage{titlesec}
\usepackage{everyshi}% http://ctan.org/pkg/everyshi
\usepackage{lipsum}
\usepackage{enumerate}
\usepackage{tabularx}


% definitions
\newcounter{weekpage}

% everyshi options
\EveryShipout{\stepcounter{weekpage}}% Step pagecntr every page



% titlesec options
\titleformat{\chapter}[display]
  {\normalfont\bfseries}{\Large Math 300: Advanced Boundary Value Problems}{10pt}{\huge}


% fancyhrd options
\setlength{\headheight}{15pt}
\pagestyle{fancy}

\renewcommand{\chaptermark}[1]{ \markboth{#1}{} }
\renewcommand{\sectionmark}[1]{ \markright{#1}{} }

\fancypagestyle{plain}{%
    \fancyhf{}
    \fancyfoot[C]{\thepage}
    \renewcommand{\headrulewidth}{0pt}
}

\fancypagestyle{plainweek}{%
    \fancyhf{}
    \fancyfoot[C]{\bf \thepage~(\thechapter.\theweekpage)} % two page counters per week
%     \fancyfoot[C]{\bf \theweekpage} % only one page counter per week
    \renewcommand{\headrulewidth}{0pt}
}

\lhead{\bf MATH 300}
\chead{{\bf\leftmark}}
\rhead{\rightmark}
\cfoot{\bf\thepage~(\thechapter.\theweekpage)} % two page counters per week
% \cfoot{\bf\theweekpage} % only one page counter per week


% Title
\title{{\bf MATH 300: Advanced \\ Boundary Value Problems I} \\
Lecture notes}
\author{Carlos Contreras}
\date{University of Alberta \\ Spring/Summer 2020}





\begin{document}


\thispagestyle{empty}

\frontmatter

\maketitle

\tableofcontents


\chapter{Preface}
This is a notebook for complementary notes to  fill out during lectures splited in thidteen weeks. The content is taking from the book ``Partial Differential Equations:
Theory and Completely Solved
Problems'' by T. Hillen, I. E. Leonard, H. van Roessel, 2019, Friesen Press. Most of the text is taking verbatim from this book, but some modification have been made in some places for simplicity. 


\mainmatter


\chapter{Week 1}
\setcounter{weekpage}{1}
\thispagestyle{plainweek}


\section{Introduction}


\begin{enumerate}


\item Notation and definitions:

\begin{itemize}

\item The \textbf{patial derivative} of $f$ with respect to $x$ is denoted
\[\frac{\partial f}{\partial x} = \frac{\partial}{\partial x}f = f_{x}\]

\item \textbf{Gradient} of $f(x,y,z)$
\[\nabla f (x,y,z) = (f_{x}(x,y,z),f_{y}(x,y,z),f_{z}(x,y,z)).\]

\item \textbf{Laplacian} of $f(x,y,z)$
\[\Delta f (x,y,z)= f_{xx}(x,y,z)+f_{yy}(x,y,z)+f_{zz}(x,y,z).\]

\item \textbf{Partial differential equation (PDE)} for unknown $u(x,y)$
\[F(x,y,u,u_{x},u_{y},u_{xx},u_{xy},u_{yy},u_{xxx},\cdots)=0.\]

\item \textbf{Linear differential operator} $L$ satisfies
\[L(u+v)=Lu+Lv \quad \text{and} \quad L(\lambda u) = \lambda u.\]

\item \textbf{Linear PDE} for unknown $u$
\[Lu=f,\]
where $L$ is a linear differential operator and function $f$ does not depend on $u$ or any of its derivatives. The equation is \textbf{homogeneous} if $f=0$, and \textbf{nonhomogeneous} if $f\neq0$.

\item The \textbf{order of a PDE} is the highest order derivative in the equation.
\end{itemize}


\newpage 
\item \textit{Example 1.1.}

Find the dimension and order of the following PDEs. Which are linear, and which are homogeneous?

\begin{itemize}
    \item {Heat equation:} 
    \[u_{t} = D u_{xx} + f(x)\]
    \vspace{20pt}

    \item {Wave equation:}
    \[u_{tt} - c u_{xx} =0 \]
    \vspace{20pt}

    \item {Laplace equation:}
    \[u_{xx} + u_{yy} = 0 \]
    \vspace{20pt}

    \item {Advection equation:}
    \[ \frac{\partial u}{\partial x}+ \frac{\partial u}{\partial y} = 0\]
    \vspace{20pt}

    \[ \frac{\partial^{2} u}{\partial x^{2}}+ e^{y}\sin(z)\frac{\partial^{2} u}{\partial x \partial z} = u\]
    \vspace{40pt}

    \[ \frac{\partial^{2} u}{\partial x \partial y} = \sin(u)\]
    \vspace{20pt}
    \item {KdV equation:} \[u_{t} + uu_{xx} + u_{xxx} =1\]
    \vspace{20pt}
\end{itemize}


\newpage

\item The \textbf{second-order linear constant-coefficiens homogeneous PDEs}
\[au_{xx}+2bu_{xy}+cu_{yy}+du_{x}+eu_{y}+fu=0\]
is said to be
\begin{itemize}
    \item \textbf{elliptic} iff $ac-b^{2}>0$.
    \item \textbf{parabolic} iff $ac-b^{2}=0$.
    \item \textbf{hyperbolic} iff $ac-b^{2}<0$.
\end{itemize}

\item \textit{Example 1.2.} 

Classify the following second-order linear PDEs.
\begin{itemize}
    \item $u _{t} + 2u _{tt} + 3u _{xx} = 0$
    \vspace{100pt}

    \item $17u_{ yy} + 3u _{x} + u = 0$
    \vspace{100pt}

    \item $4u _{xy} + 2u _{xx} + u _{yy} = 0$
    \vspace{100pt}

    \item $u _{yy} - u _{xx} - 2u _{xy} = 0$
    \vspace{100pt}

\end{itemize}



\newpage

\item \textbf{Superposition principle}.
If $u_{1}$ and $u_{2}$ are solutions to $Lu=0$, so is $c_{1}u_{1}+c_{2}u_{2}$.

\item \textbf{Theorem 1.2} If $u_{p}$ is a particular solution to $Lu=f$ and $u_{h}$ is the solutions to $L=u$, then $u=cu_{h}+u_{p}$ is a solution $Lu=f$ for any $c$.

\item \textit{Example 1.7. (Burger’s Equation)}

Consider the following two-dimensional first-order nonlinear PDE:
\[u_{x}+uu_{y}=0\]
and solutions
\[u_{1}(x,y)=1 \quad \text{and} \quad u_{2}(x,y)=\frac{y}{1+x}.\]


\vspace{200pt}

Consider the nonhomogeneous case:
\[u_{x}+uu_{y}=\frac{y^{2}-1}{x^{2}y^{3}}\]
with particular solution
\[u_{p}(x,y)=-\frac{1}{xy}.\]





\newpage

\item \textbf{Conditions}: a PDE can have
\begin{itemize}
    \item \textbf{Initial conditions}: value at time $t=0$, i.e., $u(x,y,0)=u_{0}(x,y)$.
    \item \textbf{Boundary conditions}: value on the boundary $\partial\Omega$ for all time
    \begin{itemize}
        \item Dirichlet: $u=g$ on $\partial\Omega$. Homogeneous if $g=0$.
        \item Neumann: $\frac{\partial u}{\partial n} =g$ on $\partial\Omega$. Homogeneous if $g=0$.
        \item Robin: $\alpha u +\beta \frac{\partial u}{\partial n} = g$ on $\partial\Omega$. Homogeneous if $g=0$.
    \end{itemize}

\end{itemize}


\item A \textbf{Boundary Value Problem} BVP is a PDE with boundary conditions.


\item A \textbf{steady-state solution} to a BVP does not depend on time, i.e., $u(x,t)=\tilde u (x)$.

\item \textit{Example 1.10.}

Find the steady-state solution to the following PDE on $[0, 2\pi]$ :
\begin{align*}
    & u_{ t} = 3u _{xx} + 9 \sin x, \\
    & u(x, 0) = 9 \sin x, \\
    & u(0, t) = 9, \\
    & u _{x} (2\pi, t) = 0.
\end{align*}

\newpage

\item \textbf{Exercise 15.1}

Show that the function
\[u = \frac{1}{\sqrt{x^{2}+y^{2}+z^{2}}}\]
is harmonic; that is, it is a solution to the three-dimensional Laplace equation $\Delta u = 0$.

\newpage

\item \textbf{Exercise 15.4}

Compute the Laplacian of the function
\[u(x,y)=\log\left( x^{2} +y^{2}\right)\]
in an appropriate coordinate system and decide if the given function satisfies Laplace's equation $\nabla ^{2} u = 0$.


\end{enumerate}




\chapter{Week 2}
\setcounter{weekpage}{1}
\thispagestyle{plainweek}


\section{Heat, wave and Laplace's equations}

\begin{enumerate}


%     \item \textbf{Separation of variables} is a method to solve for unkown $u(x,t)$ of two-variables
%     \[u(x,t)=X(x)T(t).\]
% This is known as \textbf{separation of variables}. After substituting into the PDEs and using the side conditions, the method leads to
% \begin{itemize}
%     \item an eigenvalue problem for $X(x)$,
%     \item an ordinary  differential equation with side conditions for $T(t)$, and
%     \item one or more Fourier series representation problems.
% \end{itemize}
% We know that eigenvalue problems usually have infinitely many solutions, which give infinitely many solutions
% \[u_{n}(x,t) = X_{n}(x)T_{n}(t).\]
% Here we use the superposition principle to obtain
% \[u_{n}(x,t)=\sum_{n}c_{n}u_{n}(x,t)=\sum_{n}c_{n}X(x)T(t).\]



\item The \textbf{heat equation} is given by
\[ u_{t} = k \Delta u + F,\]
where $k$ is the \textbf{thermal diffusivity} and $F$ is the forcing term.

\vspace{80pt}

\item The \textbf{wave equation} is given by
\[  u_{tt} = c^{2} \Delta u + F,\]
where $c$ is the \textbf{velocity of wave propagation} and $F$ is the forcing term.


\vspace{80pt}


\item \textbf{Laplace's equation}, also \textbf{potential equation} is given by
\[ \Delta u = 0.\]


\vspace{80pt}

\textbf{Poisson's equation} is the nonhomogeneous version
\[ \Delta u = F,\]
where $F$ is the forcing term and



\newpage

\item \textbf{Exercise 13.1}

For each of the boundary value problems below, determine whether or not an equilibrium temperature distribution exists and find the values of $\beta$ for which an equilibrium solution exists.
\begin{enumerate}
    \item $u_{t}=u_{xx} + 1, \quad u_{x}(0,t)=1, \quad u_{x}(a,t)=\beta.$
    \vspace{180pt}

    \item $u_{t}=u_{xx}, \quad u_{x}(0,t)=1, \quad u_{x}(a,t)=\beta.$
    \vspace{180pt}

    \item $u_{t}=u_{xx} + x - \beta , \quad u_{x}(0,t)=0, \quad u_{x}(a,t)=0.$
\end{enumerate}



\newpage 

% \item Solve the one-dimensional heat equation with Neumann boundary conditions on the interval $[0,1]$:
% \begin{align*}
%     & u_{t} = \tfrac{1}{5} u_{xx}+, 
%     \quad x\in\Omega =[0,l], \quad t>0,\\
%     & u_{t}(0,t) = 0, \\
%     & u_{t}(l,t) = 0, \\
%     & u(x,0) = f(x)=6 + 4\cos \frac{3\pi x}{l} .
% \end{align*}


\end{enumerate}



\section{Fourier Series}

\begin{enumerate}

\item \textbf{Definition 2.1}. Let the function $f$ be defined on an open interval containing the
point $x_0$.
\begin{itemize}
\item[(i)] If $f (x^{ +}_{0} ) = f (x ^{-}_{0} ) = f (x _{0} )$, $f$ is \textbf{continuous} at $x _{0}$; and \textbf{discontinuous} at $x _{0}$, otherwise.

\item[(ii)] If $f$ is discontinuous at $x _{0}$ and if both $f (x ^{+}_{0} )$ and $f (x ^{-}_{0} )$ exist, $f$ is said to have a \textbf{discontinuity of the first kind} or a \textbf{simple discontinuity} at $x _{0}$.

\item[(iii)] A simple discontinuity of $f$ of the first kind at $x _{0}$ is said to be
\begin{itemize}

    \item[(a)] a \textbf{removable discontinuity} if $f (x ^{+}
_{0} ) = f (x ^{-}_{0} ) \neq f (x _{0} )$ and

    \item[(b)] a \textbf{jump discontinuity} if $f (x ^{+}_{0} ) \neq f (x^{-} _{0} )$, regardless of the value $f (x _{0} )$.

\end{itemize}

\item[(iv)] Any discontinuity of $f$ at $x _{0}$ not of the first kind is said to be a \textbf{discontinuity of the second kind} at $x _{0}$.
\end{itemize}

\item \textbf{Definition 2.2.} A function $f$ is \textbf{piecewise continuous} (PWC) on an interval $(a, b)$ if
\begin{itemize}
\item[(i)] $f$ is continuous for $x \in (a, b)$ except possibly at a finite number of points;
\item[(ii)] $f (x ^{+} )$ exists for all $x \in [a, b)$;
\item[(iii)] $f (x ^{-} )$ exists for all $x \in (a, b]$.
\end{itemize}

Notation. $PWC(a, b)$ denotes the set of all PWC functions on $(a, b)$.

\item \textbf{Theorem 2.1.} [Properies of $PWC(a, b)$]
\begin{itemize}
\item[(i)] If $f, g \in PWC(a, b)$, then $\alpha f + \beta g ∈ PWC(a, b)$ for all $\alpha, \beta \in R$.
\item[(ii)] If $f, g \in PWC(a, b)$, then $f \cdot g \in PWC(a, b)$.
\item[(iii)] If $f \in PWC(a, b)$, then
$\int_{a}^{b}|f(x)|dx$ exists.
\end{itemize}


\item \textbf{Definition 2.3.} A function $f$ is \textbf{piecewise smooth} (PWS) on $(a, b)$ if
\begin{itemize}
\item[(i)] $f \in PWC(a, b)$ and
\item[(ii)] $f' \in PWC(a, b)$.
\end{itemize}

Notation. $PWS(a, b)$ denotes the set of all PWS functions on $(a, b)$.



\item \textit{Example}. Consider the following functions

\begin{enumerate}
    \item $
    f(x) = 
    \begin{cases}
        e^{x}, & \text{for } x\neq 1 \\
        1, & \text{for } x = 1.
    \end{cases}
    $

    \vspace{20pt}
    \item $
    g(x) = 
    \begin{cases}
        \sin(x), & \text{if } x\neq 0 \\
        0, & \text{if } x = 0.
    \end{cases}
    $

    \vspace{20pt}
    \item $
    h(x) = 
    \begin{cases}
        x, &  0 < x \leq 1 \\
        -1, &  1 < x \leq 2 \\
        1, & 2 < x < 3.
    \end{cases}
    $
\end{enumerate}


\newpage 

\item \textbf{Definition 2.4.} Let $f$ be a function whose domain $D(f)$ is symmetric, that is, $-x \in D(f )$ whenever $x \in D(f )$; then we say that 
\begin{itemize}
\item[(i)] $f$ is \textbf{even} if $f (-x) = f (x) $ for all $x \in D(f )$.
\item[(ii)] $f$ is \textbf{odd} if $f (-x) = -f (x)$ for all $x \in D(f )$.
\item[(iii)] $f$ is \textbf{periodic} with period $p$ if $x + p \in D(f )$ whenever $x \in D(f )$, and
$f (x + p) = f (x)$ for all $x \in D(f )$.
\end{itemize}


\item The \textbf{periodic extension} of $f$ defined on $(a,b)$, denoted $\bar f$, is defined as
\[\bar f(x)=f(x,np) \quad \text{for} \quad a-np<x<b-np, \quad n \in \mathbb{Z}. \]


\item \textbf{Definition 2.5}. If the function $f$ is defined on the interval $(0, l)$:

\begin{itemize}
\item[(i)] The \textbf{odd extension} of $f$ on $(-l,l)$, denoted $f_{\text{odd}}$, is defined by
\begin{equation*}
    f_{\text{odd}}(x)=
    \begin{cases}
        f(x), & \text{for} \quad 0<x<l, \\
        -f(-x), & \text{for} \quad -l<x<0,
    \end{cases}
\end{equation*}
and

\item[(ii)] The \textbf{even extension} of $f$ on $(-l,l)$, denoted $f_{\text{even}}$, is defined by
\begin{equation*}
    f_{\text{even}}(x)=
    \begin{cases}
        f(x), & \text{for} \quad 0<x<l, \\
        f(-x), & \text{for} \quad -l<x<0.
    \end{cases}
\end{equation*}

\end{itemize}

% \item \textbf{Definition 2.6.} An \textbf{inner product} on a vector space $X$ is any function $\langle u, v\rangle$ that acts on pairs of vectors $u$ and $v$ in $X$ and satisfies the following properties:
% 
% For any $u, v, w \in X$ and $\lambda\in R$ :
% 
% \begin{itemize}
% \item[(i)] $\langle u, u\rangle \geq 0$ and $\langle u, u\rangle = 0 $ if and only if $u = 0$,
% \item[(ii)] $\langle u, v\rangle = \langle v, u\rangle$ ,
% \item[(iii)] $\langle u, v + w\rangle = \langle u, v\rangle + \langle u, w\rangle$, and
% \item[(iv)] $\langle \lambda u, v\rangle = \lambda \langle u, v\rangle$.
% \end{itemize}

\item \textbf{Definition 2.7}. Let $f, g, w \in PWC(a, b)$ with $w(x) \geq 0$. The \textbf{inner product}
of $f$ and $g$ with \textbf{weight function} $w$ is defined as
\[\langle f,g \rangle = \int_{a}^{b}f(x)g(x)w(x)dx.\]


\item \textbf{Definition 2.8}. The \textbf{norm} of $f \in P W C(a, b)$ with weight $w$ is $\|f\|=\sqrt{\langle f,f \rangle}$.

\item \textbf{Definition 2.9}. If $f, g, w \in P W C(a, b)$ with \textbf{weight function} $w(x) \geq 0$, $f$ and $g$ are said to be \textbf{orthogonal} on $(a, b)$ relative to the weight $w$ if $\langle f, g\rangle =0$.
% \[\langle f, g\rangle = \int_{a}^{b}f(x)g(x)w(x)dx =  0.\]




\item The set 
\[\left\{ 1, \cos \frac{\pi x}{l}, \sin\frac{\pi x}{l} , \cos\frac{2\pi x}{l} , \sin\frac{2\pi x}{l} , \cos\frac{3\pi x}{l}, \sin\frac{3\pi x}{l} , \dots \right\} \]
is an \textbf{orthogonal set of functions} on $(a, b)$ with respect to the inner product above, where $l=(b-a)/2$.




\newpage 

\item \textbf{Exercise 11.3} 

Evaluate
\[\int_{0}^{a}\cos\frac{n\pi x}{a}\cos\frac{m\pi x}{a}dx\]
for $n \geq 0 $, $m \geq 0$. Use the trigonometric identity
\[\cos A \cos B = \frac{1}{2}[\cos(A+B) + \cos(A-B)]\]
consider $A-B=0$ and $A+B=0$ separately.

\vspace{340pt}

\item \textbf{Exercise 11.4}

Evaluate
\[\int_{0}^{a}\sin\frac{n\pi x}{a}\sin\frac{m\pi x}{a}dx\]
for $n \geq 0 $, $m> 0$ and consider $n=m$ separately. Use the trigonometric identity
\[\sin A \sin B = \frac{1}{2}[\cos(A-B) - \cos(A+B)].\]



\newpage

\item \textbf{Definition 2.10}.  The \textbf{Fourier series}  of $f$ on $(a, b)$ is given by
\[f(x) \sim a_{0}+\sum_{n=1}^{\infty}a_{n}\cos \frac{n\pi x}{l} + b_{n}\sin \frac{n\pi x}{l},\]
where $l=(b-a)/2$ and
\[a_{0} = \frac{1}{2l}\int_{a}^{b}f(x)dx, \quad a_{n} = \frac{1}{l}\int_{a}^{b}f(x)\cos\frac{n\pi x}{l}dx,\quad b_{n} = \frac{1}{l}\int_{a}^{b}f(x)\sin\frac{n\pi x}{l}dx,\quad n\geq 1,\]
are called the \textbf{Fourier coefficients} of $f$.

\item \textit{Example 2.8}.

Find the Fourier series for the $2\pi$-periodic function $f$ defined by
\[f(x)=
\begin{cases}
x & 0<x<\pi ,\\
0 & -\pi<x<0,
\end{cases}
\]
and $f(x+2\pi)=f(x)$ otherwise.





\newpage



\item \textbf{Theorem 2.2}. For $f \in P W C(-l, l)$, the following are true:

\begin{itemize}
    \item[(a)] If $f$ is an odd function,
    \[f(x) \sim \sum_{n=1}^{\infty}b_{n}\sin\frac{n\pi x}{l};\]
that is, the Fourier series for $f$ contains only sine terms.
    \item[(b)] If $f$ is an even function,
    \[f(x) \sim a_{0} + \sum_{n=1}^{\infty}a_{n}\cos\frac{n\pi x}{l};\]
that is, the Fourier series for $f$ contains only cosine terms.
\end{itemize}



\item Let function $f$ defined on $(0,l)$.
\begin{itemize}
    \item[(i)] The \textbf{Fourier sine series} for $f$ is
    \[f(x) \sim \sum_{n=1}^{\infty}b_{n}\sin\frac{n\pi x}{l},\]
    where
    \[b_{n} = \frac{2}{l}\int_{0}^{l}f(x)\sin\frac{n\pi x}{l}dx \quad \text{for} \quad n \geq 1. \]
    Note that this defines $f_{\text{even}}$, the  odd extension of $f$ on $(-l,l)$.
    \item[(ii)] The \textbf{Fourier cosine series} for $f$ is
    \[f(x) \sim a_{0} + \sum_{n=1}^{\infty}a_{n}\sin\frac{n\pi x}{l},\]
    where
    \[a_{0} = \frac{1}{l}\int_{0}^{l}f(x)dx, \quad \text{and} \quad a_{n} = \frac{2}{l}\int_{0}^{l}f(x)\cos\frac{n\pi x}{l}dx\quad \text{for} \quad n \geq 1. \]
    Note that this defines $f_{\text{even}}$, the even extension of $f$ on $(-l,l)$.
\end{itemize}

\newpage

\item \textit{Example 2.10a}. Find the Fourier sine series of the function
\[
f(x)=
\begin{cases}
    2x, & 0<x<1, \\
    2, & 1<x<2.
\end{cases}
\]


\newpage

\item \textit{Example 2.10b}. Find the Fourier cosine series of the function
\[
f(x)=
\begin{cases}
    2x, & 0<x<1, \\
    2, & 1<x<2.
\end{cases}
\]

\newpage



\item \textbf{Excercise 18.2}

Let $f(x)=\cos^{2}(x), 0 < x < \pi$.

\begin{enumerate}
\item Find the Fourier sine series for $f$ on the interval $(0,\pi)$.

Hint: For $n\geq 1$,
\[\int \cos^{2}x\sin nx dx = -\frac{1}{2n}\cos nx + \frac{1}{4}\int [\sin (n+2)x + \sin(n-2)x ]dx.  \]
\item Find the Fourier cosine series for $f$ on the interval $(0,\pi)$.
\end{enumerate}




\end{enumerate}




\chapter{Week 3}
\setcounter{weekpage}{1}
\thispagestyle{plainweek}

\section{Fourier Series}

\begin{enumerate}




\item \textbf{Theorem 2.3}. \textit{(Dirichlet's Theorem)}

Let $f (x)$ be piecewise smooth on the interval $(−l, l)$. The Fourier series 
\[a_{0}+\sum_{n=1}^{\infty}a_{n}\cos \frac{n\pi x}{l} + b_{n}\sin \frac{n\pi x}{l},\]
where
\[a_{0} = \frac{1}{2l}\int_{-l}^{l}f(x)dx, \quad a_{n} = \frac{1}{l}\int_{-l}^{l}f(x)\cos\frac{n\pi x}{l}dx,\quad b_{n} = \frac{1}{l}\int_{-l}^{l}f(x)\sin\frac{n\pi x}{l}dx,\quad n\geq 1,\]
has the following properties:
\begin{itemize}

\item[(i)] If $f (x)$ is continuous at $x _{0}$ , where $-l < x _{0} < l$, then
\[f(x_{0})=a_{0}+\sum_{n=1}^{\infty}a_{n}\cos \frac{n\pi x_{0}}{l} + b_{n}\sin \frac{n\pi x_{0}}{l};\]
that is, the Fourier series converges to $f (x _{0} )$.
\item[(ii)] If $f (x)$ has a jump discontinuity at $x _{0}$ , where $−l < x 0 < l$, then
\[\frac{f(x_{0}^{+})+f(x_{0}^{-})}{2} = a_{0}+\sum_{n=1}^{\infty}a_{n}\cos \frac{n\pi x_{0}}{l} + b_{n}\sin \frac{n\pi x_{0}}{l};\]
that is, the Fourier series converges to the \textbf{average} or \textbf{mean} of the jump.
\item[(iii)] At the endpoints $x 0 = \pm l$, the Fourier series converges to
\[\frac{f(-l^{+})+f(l^{-})}{2}.\]
\end{itemize}
As usual, we write
\[f(x)\sim a_{0}+\sum_{n=1}^{\infty}a_{n}\cos \frac{n\pi x}{l} + b_{n}\sin \frac{n\pi x}{l},\]
and say that $f (x)$ is \textbf{represented by its Fourier series} on the interval $(−l, l)$.



The Fourier series defines a $2l$-periodic extension of $f(x)$ for all $x\in \mathbb{R}$.

\newpage


\item \textbf{Exercise 11.5} 

Compute the Fourier series of the $2\pi$-periodic function $f$ given by
\[
f(x)=
\begin{cases}
1, & 0<x<\pi/2, \\
0, & \pi/2<|x|< \pi, \\
-1, & -\pi/2<x<0.\\
\end{cases}
\]
For which values of $x$ does the Fourier series converge to $f$? Sketch the graph of the Fourier.





\newpage

\item \textbf{Exercise 11.6} 

Compute the Fourier series of the $2\pi$-periodic function $f$ given by $f(x)=|\cos(x)|$. For which values of $x$ does the Fourier series converge to $f$? Sketch the graph of the Fourier.


\newpage 

(continue)


\newpage 


\item \textbf{Exercise 11.7} 

Consider the parabola $f(x)=x^{2}$ on $[-a,a]$ and show that the Fourier series of $f$ is given by 
\[\frac{a^{2}}{3} - \frac{4a^{2}}{\pi^{2}} \left[ \cos\left( \frac{\pi x}{a} \right) - \frac{1}{2^{2}}\cos\left( \frac{2 \pi x}{a} \right) + \frac{1}{3^{2}}\cos\left( \frac{3 \pi x}{a} \right) +\cdots \right]. \]
Find its values and the points of discontinuity.



\newpage



\item \textbf{Theorem 2.4.} \textit{(Uniqueness of Fourier Series)} 

If $f$ is $2 l$-periodic and piecewise smooth on the interval $(-l,l)$, its Fourier series is unique.



\item \textbf{Theorem 2.5.} \textit{(Linearity of Fourier Series)}

If $f$ and $g$ are piecewise continuous on $(-l,l)$ and $c _{1}$ and $c _{2}$ are scalars, the Fourier series of
\[c_{1}f+c_{2}g\]
is the sum of $c _{1}$ times the Fourier series of $f (x)$ and $c _{2}$ times the Fourier series of $g(x)$.



\item \textbf{Theorem 2.8.} \textit{(Term-by-Term Differentiation of Fourier Series)}

Let $f$ be a function such that
\begin{itemize}
\item[(i)] $f$ is continuous on the interval $-\pi \leq x \leq \pi$;
\item[(ii)] $f (-\pi) = f (\pi)$; and
\item[(iii)] $f'$ is piecewise smooth on the interval $-\pi < x < \pi$.
\end{itemize}

% The Fourier series representation
% \[f(x) = a_{0}+\sum_{n=1}^{\infty}a_{n}\cos nx + b_{n}\sin nx,\]
% is differentiable at each point $x _{0}$ with $-\pi < x _{0} < \pi$ at which $f''(x _{0} )$
% exists, and
% \[f'(x_{0}) = \sum_{n=1}^{\infty}n(-a_{n}\sin nx_{0} + b_{n}\cos nx_{0}).\]
% At a point $x _{0}$ with $-\pi < x _{0} < \pi$ at which $f '' (x _{0} )$ does not exist but where $f'$ has one-sided derivatives, the series above converges to 
% \[\frac{f'(x_{0}^{+})+f'(x_{0}^{-})}{2}.\]
The derivative of the Fourier series representation of $f$ is represented by
\[
f'(x) \sim
\begin{cases}
\displaystyle \sum_{n=1}^{\infty}n(-a_{n}\sin nx_{0} + b_{n} \cos nx_{0}), & \text{if  $f''(x _{0} )$
exists} \\
\displaystyle \frac{f'(x_{0}^{+})+f'(x_{0}^{-})}{2}, & \text{if $f '' (x _{0} )$ DNE but one-sided derivatives exist.}
\end{cases}
\]


\item \textbf{Theorem 2.9.} \textit{(Term-by-Term Differentiation of Fourier Cosine Series)}

Let $f$ be a function such that
\begin{itemize}
 \item[(i)] $f$ is continuous on the interval $0 \leq x \leq \pi$;
 \item[(ii)] $f'$ is piecewise continuous on the interval $0 < x < \pi$.
\end{itemize}
% The Fourier cosine series representation
% \[f(x) = a_{0}+\sum_{n=1}^{\infty}a_{n}\cos nx,\]
% is differentiable at each point $x _{0}$ with $0 < x _{0} < \pi$ at which $f''(x _{0} )$
% exists, and
% \[f'(x_{0}) = -\sum_{n=1}^{\infty}na_{n}\sin nx_{0}.\]
% At a point $x _{0}$ with $-\pi < x _{0} < \pi$ at which $f '' (x _{0} )$ does not exist but where $f'$ has one-sided derivatives, the series above converges to 
% \[\frac{f'(x_{0}^{+})+f'(x_{0}^{-})}{2}.\]
The derivative of the Fourier Cosine series representation of $f$ is represented by
\[
f'(x) \sim
\begin{cases}
\displaystyle -\sum_{n=1}^{\infty}na_{n}\sin nx_{0} , & \text{if  $f''(x _{0} )$
exists} \\
\displaystyle \frac{f'(x_{0}^{+})+f'(x_{0}^{-})}{2}, & \text{if $f '' (x _{0} )$ DNE but one-sided derivatives exist.}
\end{cases}
\]

\item \textbf{Theorem 2.10.} \textit{(Term-by-Term Differentiation of Fourier Sine Series)}

Let $f$ be a function such that
\begin{itemize}
\item[(i)] $f$ is continuous on the interval $0 \leq x \leq \pi$;
\item[(ii)] $f (0) = f (\pi)$; and
\item[(iii)] $f'$ is piecewise smooth on the interval $0 < x < \pi$.
\end{itemize}

% The Fourier sine series representation
% \[f(x) = \sum_{n=1}^{\infty} b_{n}\sin nx,\]
% is differentiable at each point $x _{0}$ with $0 < x _{0} < \pi$ at which $f''(x _{0} )$
% exists, and
% \[f'(x_{0}) = \sum_{n=1}^{\infty}n(-a_{n}\sin nx_{0} + b_{n}\cos nx_{0}).\]
% At a point $x _{0}$ with $-\pi < x _{0} < \pi$ at which $f '' (x _{0} )$ does not exist but where $f'$ has one-sided derivatives, the series above converges to 
% \[\frac{f'(x_{0}^{+})+f'(x_{0}^{-})}{2}.\]
The derivative of the Fourier Sine series representation of $f$ is represented by
\[
f'(x) \sim
\begin{cases}
\displaystyle \sum_{n=1}^{\infty}nb_{n}\cos nx_{0} , & \text{if  $f''(x _{0} )$
exists} \\
\displaystyle \frac{f'(x_{0}^{+})+f'(x_{0}^{-})}{2}, & \text{if $f '' (x _{0} )$ DNE but one-sided derivatives exist.}
\end{cases}
\]



\item \textbf{Theorem 2.11.} \textit{(Term-by-Term Integration of Fourier Series)}

Let $f$ be piecewise continuous on the interval $-\pi < x < \pi$, and suppose that on $(-\pi,\pi)$
\[f(x) \sim a_{0}+\sum_{n=1}^{\infty}a_{n}\cos nx + b_{n}\sin nx,\]
then for $-\pi\leq x \leq \pi$
\[\int_{-\pi}^{\pi}f(t)dt = a_{0}(x+\pi) + \sum_{n=1}^{\infty}\frac{1}{n}\{a_{n}\sin nx - b_{n}[(-1)^{n+1} + \cos nx]\}.\]




\item \textbf{Exercise 11.8} 

Consider the $2a$-periodic function $f$ that is given on the interval $-a < x < a$ by $f (x) = x$. Show that the Fourier series of $f$ is given by

\[\frac{2a}{\pi}\sum_{n=1}^{\infty}\frac{(-1)^{n+1}}{n}\sin\left( \frac{n\pi x}{a} \right)\]
by differentiating the Fourier series in \textit{Exercise 11.7} term-by-term. Justify your work.



\newpage

\item \textbf{Euler's formula} in complex variables
\[e^{i\theta} = \cos \theta + i\sin \theta,\]
and complex trigonometric formulas
\[\cos \theta = \frac{e^{i\theta} + e^{-i\theta} }{2}= \cosh i\theta  \qquad \text{and} \qquad \sin \theta = \frac{e^{i\theta} - e^{-i\theta} }{2}=-i \sinh i\theta.\]



\item \textbf{Theorem 2.14.} The complex Fourier series for $f \in P W C(-l, l)$ is 

\[f(x) \sim \sum _{n=-\infty}^{\infty} c_{n}e^{i n\pi x/ l }, \qquad \text{where} \qquad c_{n}=\frac{1}{2l}\int_{-l}^{l}f(x)e^{-i n \pi x/l}dx, \quad n\in \mathbb{Z}.\]



\item \textit{Example 2.19.} Calculate the complex Fourier series for
\[f(x)=x, \quad -\pi<x<\pi,\]
and $f(x+2\pi)=f(x)$ otherwise.

\newpage

\null



\end{enumerate}





\chapter{Week 4}
\setcounter{weekpage}{1}
\thispagestyle{plainweek}

\section[Separation of Variables]{Separation of Variables: Homogeneous equations}

\begin{enumerate}



\item In the method of \textbf{separations of variables} we look for solutions of the form
\[u(x,t)=X(x)T(t).\]


\item The \textbf{eigenvalue problem with homogeneous Dirichlet boundary conditions}
\[X''+\lambda X = 0 \, \quad X(0)=0, \quad X(l)=0,\]
has nontrivial solution for eigenvalues and corresponding eigenfunctions
\[\lambda_{n} = \left(\frac{n\pi}{l}\right)^{2}, \quad  X_{n}(x)= \sin \frac{n\pi x}{l}, \quad n\geq 1.\]


\item The \textbf{eigenvalue problem with homogeneous Neumann boundary conditions}
\[X''+\lambda X = 0 \, \quad X'(0)=0, \quad X'(l)=0,\]
has nontrivial solution for eigenvalues and corresponding eigenfunctions
\[\lambda_{n} = \left(\frac{n\pi}{l}\right)^{2}, \quad  X_{n}(x)= \cos \frac{n\pi x}{l}, \quad n\geq 0.\]


\newpage

\item \textbf{Exercise 13.2}

Solve the homogeneous Dirichlet problem for the heat equation
\[\frac{\partial u}{\partial t} = k \frac{\partial^{2}u}{\partial x^{2}}, \quad 0< x < a, \quad t>0,\]
subject to the boundary conditions
\[u(0,t)=0, \quad \text{and} \quad u(a,t)=0,\]
for $t > 0$, with initial conditions
\[u(x,0)=
\begin{cases}
    1, & 0<x<\frac{a}{2} \\
    2, & \frac{a}{2} \leq x< a.
\end{cases}
\]


\newpage


\item \textbf{Exercise 13.3}

Solve the following boundary value--initial value problem for the heat equation
\[\frac{\partial u}{\partial t} = k \frac{\partial^{2}u}{\partial x^{2}},\]
\[u(0,t)=u(a,t)=0,\]
\[u(x,0)=3 \sin \frac{\pi x}{a} - \sin \frac{3\pi x}{a}
\]
for $0<x<a$, $t > 0$.


\newpage

\item \textbf{Exercise 14.2}

Solve the following boundary value--initial value problem for the wave equation:

\[
\begin{aligned}
    & \frac{\partial ^{2}u}{\partial t^{2}} =  \frac{\partial^{2}u}{\partial x^{2}}, \quad 0<x<1, \quad t>0 \\
    & u(0,t)=0, \\
    & u(1,t)=0, \\
    & u(x,0)=\sin \pi x + \tfrac{1}{2}\sin 3\pi x + 3 \sin 7 \pi x, \\
    & \frac{\partial u}{\partial t}(x,0)= 1.
\end{aligned}
\]

You can use the fact the $\left\{ \sin\frac{(2n+1)x}{2} \right\}_{n\geq0}$ are orthogonal on $[0,\pi]$.

\newpage 

\textit{(continue Exercise 14.2)}


\newpage

\item \textbf{Exercise 13.8}

Solve the problem of heat transfer in a bar of length $a = \pi$ and thermal diffusivity $k = 1$, with initial heat distribution $u(x, 0) = \sin x$, where one end of the bar is kept at a constant temperature $u(0, t) = 0$, while there is no heat loss at the other end of the bar, so that $\partial u(\pi, t)/\partial x = 0$, that
is, solve the boundary value--initial value problem
\[
\begin{aligned}
    & \frac{\partial u}{\partial t} = k  \frac{\partial^{2}u}{\partial x^{2}}, \quad 0<x<\pi, \quad t>0 \\
    & u(0,t)=0, \\
    & \frac{\partial u}{\partial x}u(\pi,t)=0, \\
    & u(x,0)=\sin x.
\end{aligned}
\]

\newpage 

\textit{(continue Exercise 13.8)}

\newpage 

\textit{(continue Exercise 13.8)}

\end{enumerate}





\chapter{Week 5}
\setcounter{weekpage}{1}
\thispagestyle{plainweek}

\section[Separation of Variables]{Separation of Variables: Nonhomogeneous equations}

\begin{enumerate}


\item Standard homogeneous Heat and Wave equations

\end{enumerate}


\begin{minipage}{\linewidth}
    \begin{minipage}{0.5\linewidth}
        \textbf{Heat eq. with Dirichlet BCs}
        \[
        \begin{aligned}
            & u_{t} = k  u_{xx}, \quad 0<x<a , \quad t>0, \\
            & u(0,t)=0, \quad t>0,\\
            & u(a,t)=0, \quad t>0, \\
            & u(x,0)=f(x), \quad 0<x<a.
        \end{aligned}
        \]
        The solution has the form
        \[u(x,t)=\sum_{n=1}^{\infty} b_{n}e^{-\left(\frac{n\pi}{a}\right)^{2}kt}\sin \frac{n\pi x}{a}.\]
    \end{minipage}
    \begin{minipage}{0.5\linewidth}
        \textbf{Heat equation with Neumann BCs}
        \[
        \begin{aligned}
            & u_{t} = k  u_{xx}, \quad 0<x<a , \quad t>0, \\
            & u_{x}(0,t)=0, \quad t>0,\\
            & u_{x}(a,t)=0, \quad t>0, \\
            & u(x,0)=f(x), \quad 0<x<a.
        \end{aligned}
        \]
        The solution has the form
        \[u(x,t)=a_{0}+\sum_{n=1}^{\infty} a_{n}e^{-\left(\frac{n\pi}{a}\right)^{2}kt}\cos \frac{n\pi x}{a}.\]
    \end{minipage} \\
    
    \vspace{20pt}
    \begin{minipage}{0.5\linewidth}
        \textbf{Wave equation with Dirichlet BCs}
        \[
        \begin{aligned}
            & u_{tt} = c^{2}  u_{xx}, \quad 0<x<a , \quad t>0, \\
            & u(0,t)=0, \quad t>0,\\
            & u(a,t)=0, \quad t>0, \\
            & u(x,0)=f(x), \quad 0<x<a, \\
            & u_{t}(x,0)=g(x), \quad 0<x<a.
        \end{aligned}
        \]
        The solution has the form
        \begin{multline*}
            u(x,t)=\\ 
            \sum_{n=1}^{\infty} \left(a_{n}\cos \frac{n\pi c t}{a} + b_{n}\sin \frac{n\pi c t}{a} \right) \sin \frac{n\pi x}{a}.
        \end{multline*}
    \end{minipage}
    \begin{minipage}{0.5\linewidth}
        \textbf{Wave equation with Neumann BCs}
        \[
        \begin{aligned}
            & u_{tt} = c^{2}  u_{xx}, \quad 0<x<a , \quad t>0, \\
            & u_{x}(0,t)=0, \quad t>0,\\
            & u_{x}(a,t)=0, \quad t>0, \\
            & u(x,0)=f(x), \quad 0<x<a, \\
            & u_{t}(x,0)=g(x), \quad 0<x<a.
        \end{aligned}
        \]
        The solution has the form
        \begin{multline*}
            u(x,t)= a_{0} + \\ 
            \sum_{n=1}^{\infty} \left(a_{n}\cos \frac{n\pi c t}{a} + b_{n}\sin \frac{n\pi c t}{a} \right) \cos \frac{n\pi x}{a}.
        \end{multline*}
    \end{minipage}
\end{minipage}

\newpage

\begin{enumerate}

\setcounter{enumi}{1}

\item Method for nonhomogeneous equations. Consider a solution of the form
\[u(x,t)=v(x)+w(x,t).\]



\item \textbf{Exercise 13.3 (modified)}

Solve the following boundary value--initial value problem for the heat equation
\[\frac{\partial u}{\partial t}= k\frac{\partial ^{2}u}{\partial t^{2}}+
\sin\frac{\pi x}{a},\]
\[u(0,t)=0 \quad \text{and} \quad u(a,t)=1,\]
\[u(x,0)=3\sin\frac{\pi x}{a} -\sin \frac{3\pi x}{a},\]
for $0<x<a$, $t>0$.


\newpage

\textit{(continue Exercise 13.3 (modified))}

\vfill


\item \textbf{Method of Eigenfunc tion Expansions}. Consider a solution of the form
\[u(x,t)=v(x,t)+w(x,t).\]


\vspace{100pt}


\end{enumerate}

\newpage

\section{Method of Characteristics}



\begin{enumerate}


\item \textbf{Method of Characteristics}

Consider the first-order linear time-dependent problem of the form
\[
\begin{aligned}
    & \frac{\partial u}{\partial t} + B(x,t)\frac{\partial u}{\partial x} = C(x,t,u) , \quad -\infty<x<\infty, \quad t>0, \\
    & u(x,0)=f(x).
\end{aligned}
\]
The method of characteristic consists on solving the \textbf{characteristc equations}
\[
\begin{aligned}
    & \frac{d x}{d t} = B(x,t),\\
    & \frac{d u}{d t} = C(x,t,u),
\end{aligned}
\]
and then using the initial condition.


\item \textit{Example 10.2}

Solve the following PDE for $u(x, t)$ on $-\infty<x<\infty$

\[\frac{\partial u}{\partial t} + \alpha \frac{\partial u}{\partial x} + \beta u = 0,\]
\[u(x,0)=f(x),\]
using the method of characteristics.


\newpage

\item \textbf{Exercise 17.5}

Solve the first-order equation
\[\frac{\partial u}{\partial t} + 3x \frac{\partial u}{\partial x} = 2t, \quad -\infty<x<\infty, \quad t>0,\]
\[u(x, 0) = \log(1 + x^{ 2} ).\]


\newpage


\item \textbf{Exercise 17.6}

Using the method of characteristics, solve
\[\frac{\partial w}{\partial t} + c \frac{\partial w}{\partial x} = e^{2x}, \quad -\infty<x<\infty, \quad t>0,\]
\[w(x, 0) = f(x).\]

\newpage


\item \textbf{Exercise 17.7}

Using the method of characteristics, solve
\[\frac{\partial w}{\partial t} + t \frac{\partial w}{\partial x} = 1, \quad -\infty<x<\infty, \quad t>0,\]
\[w(x, 0) = f(x).\]


\newpage


\newpage

\item \textbf{Method of Characteristics (revised)}

Consider the first-order linear problem of the form
\[
\begin{aligned}
    & A(x,y)\frac{\partial u}{\partial x} + B(x,y)\frac{\partial u}{\partial y} = C(x,y,u) , \quad -\infty<x<\infty, \quad t>0, \\
    & u(x,y)=f(x,y), \quad (x,y)\in \Gamma_{a},
\end{aligned}
\]
where $\Gamma_{a}$ is a curve of anchor points and $f$ is a given function.

Consider the surface $z=u(x,y)$ with parametrization
\[x=x_{0}(a),\quad y=y_{0}(a), \quad z=z_{0}(a)=f(x_{0}(a), y_{0}(a)).\]

This defines \textbf{characteristc equations}
\[
\begin{aligned}
    & \frac{d x}{d s} = A(x,y),\\
    & x(0)=x_{0}(a),
\end{aligned}
\]
\[
\begin{aligned}
    & \frac{d y}{d s} = B(x,y),\\
    & y(0)=y_{0}(a),
\end{aligned}
\]
\[
\begin{aligned}
    & \frac{d z}{d s} = C(x,y,z),\\
    & z(0)=z_{0}(a),
\end{aligned}
\]
which can be used to 
\begin{enumerate}
    \item Solve the first two characteristic equations to get $x$ and $y$ in terms of the characteristic variable $s$ and the anchor point $a$:
    \[x=X(s,a), \quad y=Y(s,a)\]
    \item Insert the solution from the previous step into the third characteristic equation, and solve
    the resulting equation for $z$:
    \[z=Z(s,a).\]
    \item Write the characteristic variables and anchor point $a$ in terms of the original independent variables $x$ and $y$; that is, invert
    \[x=X(s,a), \quad y=Y(s,a)\]
    to get
    \[s=S(x,y), \quad a=\Gamma(x,y).\]
    \item Write the solution for $z$ in terms of $x$ and $y$ to get the solution to the original PDE:
    \[u(x,y)=Z(S(x,y), \Gamma(x,y)).\]
\end{enumerate}

\newpage


\textit{(extra page)}

\newpage

\textit{(extra page)}



\end{enumerate}


\chapter{Week 6}
\setcounter{weekpage}{1}
\thispagestyle{plainweek}

\section{One-dimensional Wave Equation}

\begin{enumerate}


\item Consider the one-dimensional wave equation
\[\frac{\partial^{2} u}{\partial t^{2}} = c^{2} \frac{\partial^{2} u}{\partial x^{2}}, \quad -\infty<x<\infty, \quad t>0, \quad u(x,0)=f(x), \quad \frac{\partial u}{\partial t}(x,0)=g(x).\]
\textbf{d'Alembert's solution} is given by
\[u(x,t)= \frac{1}{2}[f(x+ct)+f(x-ct)] + \frac{1}{2c} \int _{x-ct}^{x+ct} g(\mu)\, d\mu.\]


\item \textbf{Exercise 17.12}

The displacement $u = u(x, t)$ of an infinitely long string is governed by the wave equation

\[\frac{\partial^{2} u}{\partial t^{2}} = 4 \frac{\partial^{2} u}{\partial x^{2}}, \quad -\infty<x<\infty, \quad t>0.\]

At time $t = 0$ an initial signal is given of the form

\[
u(x,0)=f(x)=
\begin{cases}
    x , & 0<x<1, \\
    -x+2, & 1<x<2, \\
    0, & \text{othewise},
\end{cases}
\]
\[\frac{\partial u}{\partial t}(x,0)=0, \quad -\infty<x<\infty.\]
\begin{enumerate}
    \item[a)] Solve this problem.
    \item[b)] Sketch the solution for times $t _{1}$, $t _{2}$, $t _{3}$, $t _{4}$, $t _{5}$ , with
    \[t_{1}=0, \quad 0<t_{2}<1/4, \quad t_{3}=1/4, \quad 1/4<t_{4}<1/2, \quad t_{5}=1/2.\]
    \item[c)] At what time does the signal reach the point $x = 11$?
\end{enumerate}

\newpage

\textit{(continue Exercise 17.12)}

\newpage

\textit{(continue Exercise 17.12)}

\newpage

\item Consider the one-dimensional wave equation
\[
\begin{aligned}
& \frac{\partial^{2} u}{\partial t^{2}} = c^{2} \frac{\partial^{2} u}{\partial x^{2}}, \quad 0<x<l, \quad t>0, \\
& u(0,t)=0, \quad  u(l,t)=0, \quad u(x,0)=f(x), \quad \frac{\partial u}{\partial t}(x,0)=g(x).\end{aligned}
\]
\textbf{d'Alembert's solution} is given by
\[u(x,t)= \frac{1}{2}[\bar f_{\text{odd}}(x+ct)+ \bar f_{\text{odd}}(x-ct)] + \frac{1}{2c}\int _{x-ct}^{x+ct} \bar g_{\text{odd}}(\mu)\, d\mu,\]
where $\bar f_{\text{odd}}$ and $\bar g _{\text{odd}}$ are the $2l$-periodic extension of $f$ and $g$, respectively.


\item \textbf{Exercise 14.8}

Use d'Alembert's solution to solve the boundary value--initial value problem for the wave equation:
\[
\begin{aligned}
    & \frac{\partial ^{2}u}{\partial t^{2}} =  \frac{\partial^{2}u}{\partial x^{2}}, \quad 0<x<1, \quad t>0 \\
    & u(0,t)=0, \\
    & u(1,t)=0, \\
    & u(x,0)=0, \\
    & \frac{\partial u}{\partial t}(x,0) = 1.
\end{aligned}
\]

\newpage

\item \textbf{Exercise 17.8}

Consider

\[
\begin{aligned}
& \frac{\partial u}{\partial t} + 2u \frac{\partial u}{\partial x} = 0, \quad -\infty<x<\infty, \quad t>0, \\
& u(x,t)=f(x).
\end{aligned}
\]
Show that characteristics are straight lines.



\newpage

\item \textbf{Exercise 17.9}

Consider

\[\frac{\partial u}{\partial t} + 2u \frac{\partial u}{\partial x} = 0, \quad -\infty<x<\infty, \quad t>0.\]

with

\[
u(x,0)=f(x)=
\begin{cases}
    1 , & 0<x, \\
    1+x/a, & 1<x<a, \\
    2, & x>a.
\end{cases}
\]

\begin{enumerate}
    \item[a)] Determine the equations for the characteristics. Sketch the characteristics.
    \item[b)] Determine the solution $u(x, t)$. Sketch $u(x, t)$ for t fixed.
\end{enumerate}


\newpage

\item \textbf{Exercise 14.9}

Use d'Alembert's solution to solve the boundary value--initial value problem for the wave equation:
\[
\begin{aligned}
    & \frac{\partial ^{2}u}{\partial t^{2}} =  \frac{\partial^{2}u}{\partial x^{2}}, \quad 0<x<1, \quad t>0 \\
    & u(0,t)=0, \\
    & u(1,t)=0, \\
    & u(x,0)=0, \\
    & \frac{\partial u}{\partial t}(x,0) = \sin \pi x.
\end{aligned}
\]


\newpage

\item \textbf{Exercise 14.10}

Use d'Alembert's solution to solve the boundary value--initial value problem for the wave equation:
\[
\begin{aligned}
    & \frac{\partial ^{2}u}{\partial t^{2}} =  25 \frac{\partial^{2}u}{\partial x^{2}}, \quad -\infty<x<\infty, \quad t>0 \\
    & u(x,0)=x^{2}, \\
    & \frac{\partial u}{\partial t}(x,0) = 3.
\end{aligned}
\]



\end{enumerate}

\newpage


\chapter{Week 7}
\setcounter{weekpage}{1}
\thispagestyle{plainweek}

\section{Sturm-Liouville Theory}


\begin{enumerate}


% \item The homogeneous second-order linear ODE
% \[(p(x)\phi')'+[q(x)+\lambda\sigma(x)]\phi=0, \quad a < x < b,\]
% where $p'(x)$, $q(x)$, and $\sigma(x)$ are all continuous on the interval $a < x < b$, and $p(x) > 0$ and $\sigma(x) > 0$ for $a < x < b$, is said to be in \textbf{self-adjoint form} or \textbf{Sturm-Liouville form}. We show below that this form is fairly general.
% 
% 
% 
% \item \textbf{Theorem 4.1.} Any homogeneous second-order linear ODE
% \[a _0 (x)\phi '' + a _1 (x)\phi ' + [a _2 (x) + \lambda]\phi = 0, \quad a < x < b,\]
% where the coefficient functions $a _0 (x)$, $a _1 (x)$, $a _2 (x)$ are continuous and $a _0 (x) >0$ on the interval $a < x < b$, can be put into self-adjoint form.



\item \textbf{Definition 4.1.} A \textbf{regular Sturm-Liouville problem} denotes the problem of finding an eigenfunction-eigenvalue pair $(\phi, \lambda)$ which solves the problem
\begin{align*}
& (p(x)\phi ' ) ' + [q(x) + \lambda \sigma(x)]\phi = 0, \quad a < x < b,\\
& \alpha _1 \phi(a) + \beta _1 \phi ' (a) = 0, \\
& \alpha _2 \phi(b) + \beta _2 \phi ' (b) = 0,
\end{align*}
where
\begin{enumerate}[(i)]

\item $p(x)$, $p ' (x)$, $q(x)$, and $\sigma(x)$ are real valued and continuous for $a \leq x \leq b$;

\item $p(x) > 0$ and $\sigma(x) > 0$ for $a \leq x \leq b$; and

\item $\alpha_1$ , $\alpha _2$ , $\beta _1$ , $\beta _2$ are real valued, $\alpha _1 ^2 + \beta _1 ^2 \neq 0$ and $\alpha _2 ^2 + \beta _2 ^2 \neq 0$.
\end{enumerate}



\item \textit{Example 4.2.} Consider the following boundary value problem, which we have
solved several times before:
\begin{align*}
& \phi '' + \lambda \phi = 0, \quad 0 < x < l,\\
& \phi(0) = 0, \\
& \phi(l) = 0.
\end{align*}


\newpage


\item \textbf{Definition 4.2.} A Sturm-Liouville problem is said to be \textbf{singular} if at least one of the conditions (i), (ii), or (iii) in Definition 4.1 fails, or if the interval is infinite. In the case where the interval is infinite, or one or both of the functions $p(x)$ and $\sigma(x)$ approach $0$ or $\infty$ at an endpoint of the interval, one or more of the boundary conditions are usually replaced by boundedness conditions on $\phi$.

\item \textit{Example 4.3. (Legendre's Equation)} Consider the boundary value problem for Legendre’s equation,
\begin{align*}
& ((1-x^2)\phi ' ) ' + \lambda \phi = 0, \quad -1 < x < 1,\\
& \alpha _1 \phi(-1) + \beta _1 \phi ' (-1) = 0, \\
& \alpha _2 \phi(1) + \beta _2 \phi ' (1) = 0,
\end{align*}


\vspace{150pt}


\item \textit{Example 4.4. (Bessel's Equation)} For fixed $n$, Bessel’s equation on the interval $a < r < b$,
\begin{align*}
& (r\phi ' ) ' + \left( \lambda r - \frac{n^2}{r} \right) \phi = 0,\\
& \phi(a) = 0, \\
& \phi(b) = 0,
\end{align*}



\newpage


\item \textbf{Theorem 4.2.} The spectrum of a regular Sturm-Liouville problem is a countably infinite set with no limit points, that is, an infinite discrete set.


\item \textbf{Theorem 4.3.} If $\lambda _m$ and $\lambda _n$ are distinct eigenvalues of a regular Sturm-Liouville problem, that is, $\lambda _m \neq \lambda _n$ , the corresponding eigenfunctions $\phi _m$
and $\phi _n$ are orthogonal relative to the inner product
\[\langle f, g\rangle = \int_{a}^{b} f (x)g(x) \sigma (x) dx.\]


\item \textbf{Theorem 4.4.} If $\lambda$ is an eigenvalue of a regular Sturm-Liouville problem:

\begin{enumerate}[(a)]

\item $\lambda$ is real, and

\item if $\phi$ and $\psi$ are eigenfunctions corresponding to $\lambda$,
\[\psi(x)=k\phi(x), \quad a\leq x \leq b,\]
where $k$ is a nonzero constant, and each eigenfunction can be made real-valued by multiplying it by an appropriate nonzero constant.
\end{enumerate}



\item \textit{Example 4.5. (Cauchy-Euler Equation)}
Consider the boundary value problem
\begin{align*}
    & (x\phi ')'+ \frac{\lambda}{x}\phi = 0, \quad 1<x<l, \\
    & \phi(1) = 0, \\
    & \phi(l) = 0.
\end{align*}



\newpage

\textit{(continue Example 4.5)}


% \newpage
% 
% 
% \item \textbf{Theorem 4.5.} Given the regular Sturm-Liouville problem,
% \begin{align*}
% & (p(x)\phi ' ) ' + [q(x) + \lambda \sigma(x)]\phi = 0, \quad a < x < b,\\
% & \alpha _1 \phi(a) + \beta _1 \phi ' (a) = 0, \\
% & \alpha _2 \phi(b) + \beta _2 \phi ' (b) = 0,
% \end{align*}
% with eigenvalues $\lambda_{n}$ and corresponding eigenfunctions $\phi_{n}$.
% 
% \begin{enumerate}[(a)]
%     \item The regular Sturm-Liouville problem has an infinite spectrum
%     \[S=\{\lambda_{1}, \lambda_{2}, \dots, \lambda_{n}, \dots \}\]
%     and $\lim_{n\to \infty} \lambda _{n}=+\infty$.
%     \item If $\alpha_{1}\beta_{1}\leq 0$ and $\alpha_{2}\beta_{2}\geq 0$, the spectrum is bounded below and the eigenvalues may be ordered as
%     \[\lambda_{1}<\lambda_{2}< \cdots< \lambda_{n}< \cdots.\]
%     Moreover, if $q(x) \leq 0$ for $a \leq x \leq b$, then $\lambda _{n} \geq 0$ for all $n \geq 1$.
%     \item If the eigenvalues are ordered as $\lambda_{1}<\lambda_{2}< \cdots< \lambda_{n}< \cdots$, the eigenfunction corresponding to $\lambda _{n}$ has exactly $(n - 1)$ zeros in the interval $a < x < b$.
% \end{enumerate}

\end{enumerate}

\newpage


\chapter{Week 8}
\setcounter{weekpage}{1}
\thispagestyle{plainweek}

\section{Sturm-Liouville Theory}


\begin{enumerate}



\item \textbf{Theorem 4.5.} 

Given the regular Sturm-Liouville problem,
\begin{align*}
& (p(x)\phi ' ) ' + [q(x) + \lambda \sigma(x)]\phi = 0, \quad a < x < b,\\
& \alpha _1 \phi(a) + \beta _1 \phi ' (a) = 0, \\
& \alpha _2 \phi(b) + \beta _2 \phi ' (b) = 0,
\end{align*}
with eigenvalues $\lambda_{n}$ and corresponding eigenfunctions $\phi_{n}$.

\begin{enumerate}[(a)]
    \item The regular Sturm-Liouville problem has an infinite spectrum
    \[S=\{\lambda_{1}, \lambda_{2}, \dots, \lambda_{n}, \dots \}\]
    and $\lim_{n\to \infty} \lambda _{n}=+\infty$.
    \item If $\alpha_{1}\beta_{1}\leq 0$ and $\alpha_{2}\beta_{2}\geq 0$, the spectrum is bounded below and the eigenvalues may be ordered as
    \[\lambda_{1}<\lambda_{2}< \cdots< \lambda_{n}< \cdots.\]
    Moreover, if $q(x) \leq 0$ for $a \leq x \leq b$, then $\lambda _{n} \geq 0$ for all $n \geq 1$.
    \item If the eigenvalues are ordered as $\lambda_{1}<\lambda_{2}< \cdots< \lambda_{n}< \cdots$, the eigenfunction corresponding to $\lambda _{n}$ has exactly $(n - 1)$ zeros in the interval $a < x < b$.
\end{enumerate}


% \item \textbf{Definition 4.3.} Let $f$ be piecewise continuous on the interval $[a, b]$. The eigenfunction expansion
% \[
% f(x)\sim \sum_{n}^{\infty} c_{n}\phi_{n}(x)
% \]
% with coefficients
% \[
% c_{n}=\frac{\langle f, \phi_{n}\rangle}{\|\phi_{n}\|^{2}}
% \]
% where the inner product has weight function $\sigma(x)$, is called a \textbf{generalized Fourier series of} $f$.


\item \textbf{Theorem 4.6.} \textit{(Dirichlet's Theorem)}

If $f$ is piecewise smooth on $[a, b]$, the \textbf{generalized Fourier series},
\[
f(x)\sim \sum_{n=1}^{\infty} c_{n}\phi_{n}(x), \quad 
\text{where}, \quad
c_{n}=\frac{\langle f, \phi_{n}\rangle}{\|\phi_{n}\|^{2}} = \frac{1}{\|\phi_{n}\|^{2}}\int_{a}^{b} f(x)\phi_{n}(x)\sigma(x)\,dx,
\]
for $n \geq 1$, converges pointwise to $[f (x ^{+} ) + f (x ^{-} )]/2$ for each $x \in (a, b)$.



\newpage

\item \textit{Example 4.6.} Consider the regular Sturm-Liouville problem
\begin{align*}
& \phi '' +  \lambda \phi = 0, \quad 0 < x < 1,\\
& \phi(0) = 0, \\
& 2 \phi(1) - \phi ' (1) = 0.
\end{align*}


\newpage
\textit{(continue Example 4.6)}



\newpage



\item \textit{Example 4.7.} Consider the regular Sturm-Liouville problem
\begin{align*}
& \phi '' +  \lambda^{2} \phi = 0, \quad 0 < x < \pi,\\
& \phi'(0) = 0, \\
& \phi(\pi) = 0.
\end{align*}

\begin{enumerate}[(a)]
    \item Find the eigenvalues $\lambda^{2}_{n}$ and the corresponding eigenfunctions $\phi_{n}$ for this problem.
    \item Show directly, by integration, that eigenfunctions corresponding to distinct eigenvalues are orthogonal.
    \item Given the function $f (x) = \pi^{2} - x^{2} /2, 0 < x < \pi$, find the eigenfunction expansion of $f$.
    \item Show that
    \[\frac{\pi^{3}}{32}=1 - \frac{1}{3^{3}} + \frac{1}{5^{3}} - \frac{1}{7^{3}}+ \frac{1}{9^{3}} - + \cdots. \]
\end{enumerate}



\newpage
\textit{(continue Example 4.7)}

\newpage

\item \textbf{Theorem 4.7.} 

If $(\phi _{n} , \lambda _{n})$ is an eigenpair for the regular Sturm-Liouville problem
\begin{align*}
& (p(x)\phi ' ) ' + [q(x) + \lambda \sigma(x)]\phi = 0, \quad a < x < b,\\
& \alpha _1 \phi(a) + \beta _1 \phi ' (a) = 0, \\
& \alpha _2 \phi(b) + \beta _2 \phi ' (b) = 0,
\end{align*}
then $\lambda _{n}$ can be calculated from the \textbf{Rayleigh quotient}:
\[
\lambda_{n} = \frac{ \displaystyle -p(x)\phi_{n}(x)\phi'_{n}(x)\Big|_{a}^{b} + \int_{a}^{b}(p(x)\phi'_{n}(x)^{2} - q(x)\phi_{n}(x)^{2}) \, dx }{ \displaystyle \int_{a}^{b}\phi_{n}(x)^{2}\sigma(x)\, dx}.
\]


\item \textbf{Corollary 4.1.} 

If 
\[-p(x)\phi_{n}(x)\phi'_{n}(x)\Big|_{a}^{b} = - [p(b)\phi_{n}(b)\phi'_{n}(b)-p(a)\phi_{n}(a)\phi'_{n}(a)]\geq 0,\]
and $q(x) \leq 0$ for $a < x < b$, then $\lambda _{n} > 0$.


\item The \textbf{Rayleigh quotient} for \textbf{any} PWS function $u=u(x)$ on $[a,b]$ is given by
\[
\mathcal{R}(u) = \frac{ \displaystyle -p(x)u(x)u'(x)\Big|_{a}^{b} + \int_{a}^{b}(p(x)u'(x)^{2} - q(x)u(x)^{2}) \, dx }{ \displaystyle \int_{a}^{b}u(x)^{2}\sigma(x)\, dx}.
\]

\item \textbf{Theorem 4.8.} 

Given the regular Sturm-Liouville problem
\begin{align*}
& (p(x)\phi ' ) ' + [q(x) + \lambda \sigma(x)]\phi = 0, \quad a < x < b,\\
& \alpha _1 \phi(a) + \beta _1 \phi ' (a) = 0, \\
& \alpha _2 \phi(b) + \beta _2 \phi ' (b) = 0,
\end{align*}
with spectrum
\[\lambda_{1}<\lambda_{2}<\cdots<\lambda_{n}<\cdots,\]
Then, the \textbf{leading eigenvalue} is
\[\lambda_{1}=\min_{u} \mathcal{R}(u)\]
for all continuous functions $u$ satisftying the boundary conditions
\[\alpha _1 u(a) + \beta _1 u' (a) = 0, \quad  \alpha _2 u(b) + \beta _2 u ' (b) = 0.\]


\newpage



\item \textit{Example 4.9.} Find good upper and lower bounds for the leading eigenvalue of the regular Sturm-Liouville problem
\begin{align*}
& \phi '' - x \phi + \lambda \phi = 0, \quad 0 < x < 1,\\
& \phi'(0) = 0, \\
& 2\phi(1) + \phi'(1) = 0.
\end{align*}





\newpage
\textit{(continue Example 4.9)}

\newpage


\item \textit{Example 4.10.} Find the generalized Fourier series solution to the homogeneous Neumann problem for the wave equation. Use the Rayleigh quotient to show that $\lambda _{1} > 0$.

\[
\begin{aligned}
    & \alpha(x)\frac{\partial^{2} u}{\partial t ^{2}} = \frac{\partial }{\partial x} \left( \tau(x) \frac{\partial u}{\partial x} \right) - \beta(x)u, \quad 0<x<l , \quad t>0, \\
    & \frac{\partial u}{\partial x}(0,t)=0, \quad t>0,\\
    & \frac{\partial }{\partial x}(l,t)=0, \quad t>0, \\
    & u(x,0)=f(x), \quad 0<x<l,\\
    & \frac{\partial u}{\partial t}(x,0)=g(x),
\end{aligned}
\]
where $\alpha(x) > 0$, $\tau (x) > 0$, and $\beta(x) > 0$ for $0 < x < l$.


\newpage
\textit{(continue Example 4.10)}


\end{enumerate}





\chapter{Week 9}
\setcounter{weekpage}{1}
\thispagestyle{plainweek}

\section{Sturm-Liouville Theory}


\begin{enumerate}

\item \textit{Example 4.11} Summary of standard Sturm-Liouville problems.

\begin{center}\small
\begin{tabular}{|c|c|c|c|}
    \hline
    \parbox[c][30pt]{90pt}{\centering \textbf{Model Type}} & 
    \parbox[c][30pt]{90pt}{\centering \textbf{S-L Problem}} & \parbox[t]{90pt}{\centering \textbf{Spectrum}} & 
    \parbox[c][30pt]{90pt}{\centering \textbf{Eigenfunctions}} \\
    \hline
    \parbox[c][30pt]{90pt}{\centering \textbf{Homogeneous \\ \vspace{10pt} Dirichlet B.C.}} & 
    \parbox[c][60pt]{90pt}{\centering $\phi''(x) + \lambda \phi(x)=0$ \\ \vspace{10pt} $\phi(0)=\phi(l)=0$}
    & 
    \parbox[c][60pt]{90pt}{\centering $\displaystyle \lambda_{n}=\left( \frac{n\pi}{l} \right)^{2}$ \\ \vspace{10pt} $n=1, 2, \cdots$} & 
    \parbox[c][60pt]{90pt}{\centering $\displaystyle \phi_{n} = \sin\frac{n\pi x}{l}$ \\ \vspace{10pt} $n=1, 2, \cdots$} \\
    \hline 
    \parbox[c][30pt]{90pt}{\centering \textbf{Homogeneous \\ \vspace{10pt} Neumann B.C.}} & 
    \parbox[c][60pt]{90pt}{\centering $\phi''(x) + \lambda \phi(x)=0$ \\ \vspace{10pt} $\phi'(0)=\phi'(l)=0$}
    & 
    \parbox[c][60pt]{90pt}{\centering $\displaystyle \lambda_{n}=\left( \frac{n\pi}{l} \right)^{2}$ \\ \vspace{10pt} $n=0, 1, \cdots$} & 
    \parbox[c][60pt]{90pt}{\centering $\displaystyle \phi_{n} = \cos\frac{n\pi x}{l}$ \\ \vspace{10pt} $n=0, 1, \cdots$} \\
    \hline
    \parbox[c][30pt]{90pt}{\centering \textbf{Mixed \\ \vspace{10pt} Type I}} & 
    \parbox[c][60pt]{90pt}{\centering $\phi''(x) + \lambda \phi(x)=0$ \\ \vspace{10pt} $\phi(0)=\phi'(l)=0$}
    & 
    \parbox[c][60pt]{100pt}{\centering $\displaystyle \lambda_{n}=\left( \frac{(2n-1)\pi}{2l} \right)^{2}$ \\ \vspace{10pt} $n=1, 2, \cdots$} & 
    \parbox[c][60pt]{100pt}{\centering $\displaystyle \phi_{n} = \sin\frac{(2n-1)\pi x}{2l}$ \\ \vspace{10pt} $n=1, 2, \cdots$} \\
    \hline
    \parbox[c][30pt]{90pt}{\centering \textbf{Mixed \\ \vspace{10pt} Type II}} & 
    \parbox[c][60pt]{90pt}{\centering $\phi''(x) + \lambda \phi(x)=0$ \\ \vspace{10pt} $\phi'(0)=\phi(l)=0$}
    & 
    \parbox[c][60pt]{100pt}{\centering $\displaystyle \lambda_{n}=\left( \frac{(2n-1)\pi}{2l} \right)^{2}$ \\ \vspace{10pt} $n=1, 2, \cdots$} & 
    \parbox[c][60pt]{100pt}{\centering $\displaystyle \phi_{n} = \cos\frac{(2n-1)\pi x}{2l}$ \\ \vspace{10pt} $n=1, 2, \cdots$} \\
    \hline
\end{tabular}
\end{center}


\end{enumerate}


\newpage

\section[2D Heat, Wave and Laplace Equations]{Two-Dimensional Heat, Wave and Laplace Equations}


\begin{enumerate}

\item \textbf{Exercise 14.13.}

Solve the problem for a vibrating square membrane with side length $1$, where the vibrations are governed by the following two-dimensional wave equation:
\begin{align*}
    & \frac{\partial^{2} u}{\partial t^{2}} = \frac{1}{\pi^{2}}\left(\frac{\partial^{2} u}{\partial x^{2}} + \frac{\partial^{2} u}{\partial y^{2}}\right), \quad 0<x<1 , \quad 0<y<1 , \quad t>0, \\
    & u(0,y,t) = u(1,y,t) = 0,\\
    & u(x,0,t) = u(x,1,t) = 0,\\
    & u(x,y,0) = \sin \pi x \sin \pi y, \\
    & \frac{\partial u}{\partial t}(x,y,0) = \sin \pi x. \\
\end{align*}


\newpage 
\textit{(continue Exercise 14.13)}

\newpage 
\textit{(continue Exercise 14.13)}

\newpage

\item \textbf{Heat, Wave and Laplace equations on the rectangle}

\begin{enumerate}

    \item Heat equation
    \begin{align*}
        & \frac{\partial^{2} u}{\partial t^{2}} = k\left(\frac{\partial^{2} u}{\partial x^{2}} + \frac{\partial^{2} u}{\partial y^{2}}\right), \quad 0<x<a , \quad 0<y<b , \quad t>0, \\
        & \\
        & \\
        & u(x,y,0) = f(x,y).
    \end{align*}
    \vspace{60pt}

    \item Wave equation
    \begin{align*}
        & \frac{\partial^{2} u}{\partial t^{2}} = c^{2}\left(\frac{\partial^{2} u}{\partial x^{2}} + \frac{\partial^{2} u}{\partial y^{2}}\right), \quad 0<x<a , \quad 0<y<b , \quad t>0, \\
        & \\
        & \\
        & u(x,y,0) = f(x,y), \\
        & \frac{\partial u}{\partial t}(x,y,0) = g(x,y). \\
    \end{align*}
    \vspace{60pt}

    \item Laplace equation
    \begin{align*}
        & \left(\frac{\partial^{2} u}{\partial x^{2}} + \frac{\partial^{2} u}{\partial y^{2}}\right) = 0, \quad 0<x<a , \quad 0<y<b,
    \end{align*}
    \vspace{60pt}
    

\end{enumerate}

\newpage 
\item Big picture


\end{enumerate}

\vfill

\section{Polar coordinates}

\begin{enumerate}
    \item Given a point $P$ with Cartesian coordinates $(x,y) \neq (0,0)$, the \textbf{polar coordinates} of $P$ are $(r, \theta)$, where
    \begin{align*}
        x & = r\cos \theta, \\
        y & = r\sin \theta.
    \end{align*}
    The Jacobian determinant of the transformation is
    \[\frac{\partial (x,y)}{\partial (r,\theta)} = 
    \begin{vmatrix}
        \cos \theta & - r \sin \theta \\
        \sin \theta & r \sin \theta \\
    \end{vmatrix} = r.
    \]

    \newpage

     \item The \textbf{disk of radius} $a$ is defined by 
    \[D(a)=\{(x,y) \mid x^{2}+y^{2}\leq a^{2}\} = \{(r,\theta) \mid 0\leq r \leq a, \,-\pi \leq \theta \leq \pi\}.\]

    \item The \textbf{Laplacian in polar coordinates} is
    \[\nabla^{2} u = \frac{1}{r}\frac{\partial}{\partial r}\left( r\frac{\partial u}{\partial r} \right) +\frac{1}{r^{2}}\frac{\partial^{2} u}{\partial \theta^{2}} = \frac{\partial^{2} u}{\partial r^{2}} + \frac{1}{r}\frac{\partial u}{\partial r} + \frac{1}{r^{2}}\frac{\partial^{2} u}{\partial \theta^{2}}.\]


    \item \textit{Example 6.1. (Potential in a Disk)} \textbf{Summary}. 

    The Dirichlet problem for Laplace's equation in a disk in polar coordinates is
%     \[
%     \begin{aligned}
%         & \nabla^{2} u = 0 \quad \text{on}\quad D(a)\\
%         & u(x, y) = f (x, y) \quad \text{on} \quad \partial D(a).
%     \end{aligned}
%     \]
    \[
    \begin{aligned}
        & \frac{1}{r}\frac{\partial}{\partial r}\left( r\frac{\partial u}{\partial r} \right) +\frac{1}{r^{2}}\frac{\partial^{2} u}{\partial \theta^{2}} = 0, \quad 0 < r < a, \quad -\pi < \theta < \pi, \\
        & u(r, -\pi) = u(r, \pi), \\
        & \frac{\partial u}{\partial \theta}(r, -\pi) = \frac{\partial u}{\partial \theta}(r, \pi), \\
        & \lim_{r\to 0^{+}} u(r, \theta) = u(0, \theta), \\
        & u(a, \theta) = f (\theta).
    \end{aligned}
    \]

\end{enumerate}







\chapter{Week 10}
\setcounter{weekpage}{1}
\thispagestyle{plainweek}


\section{Bessel functions}

\begin{enumerate}

    \item \textbf{Bessel's equation} of order $n$ is given by
    \[x^{2}\frac{d^{2}u}{dx^{2}}+x\frac{du}{dx}+(x^{2}-n^{2})u=0.\]

    \item The general solution of Bessel's equations of order $n$ are given by 
    \[u(x) = A \,J_{n}(x) + B\,Y_{n}(x),\]
    for arbitrary constants $A$ and $B$, where
    \[J_{n}(x) = \sum_{k=0}^{\infty}\frac{(-1)^{k}}{k!\,(n+k)!}\left( \frac{x}{2} \right)^{n+2k},\quad n\geq 0,\]
    is the \textbf{Bessel function of the first kind of order} $n$ and $Y_{n}(x)$ is the \textbf{Bessel function of the second kind of order} $n$.

%     The \textbf{Bessel function of the second kind of order} $n$ (integer)
%     \begin{multline*}
%         Y_{n}(x) = \frac{2}{\pi}\left[ J_{n(x)}\left( \gamma +\log\frac{x}{2} \right) - \frac{1}{2} \sum_{k=0}^{n-1}\frac{(n-k-1)!}{k!}\left( \frac{x}{2} \right)^{2k-n} \right. \\ 
%         \left. - \frac{1}{2} \sum_{k=1}^{\infty}\frac{(-1)^{n}[\phi(k)+\phi(k+n)]}{k!\,(n+k)!}\left( \frac{x}{2} \right)^{n-2k} \right],\quad n\geq 0,
%     \end{multline*}
%     where 
%     \[\phi(x)=1+\frac{1}{2}+\frac{1}{3}+\cdots+\frac{1}{k},\]
%     and $\gamma$ is the Euler-Mascheroni constant,
%     \[ \gamma = \lim_{k\to \infty} \left( 1+\frac{1}{2}+\frac{1}{3}+\cdots+\frac{1}{k} - \log k \right) = 0.5772157 \dots.\]

%     \item $J _{n} (x)$ is an \textbf{even} function if $n = 0, 2, 4,\dots$, and an \textbf{odd} function if $n = 1, 3, 5,\dots$.



%     \item $J_{n}(x)$ satisfies the Bessel's equation of order $n$
%     \[x^{2}\frac{d^{2}u}{dx^{2}}+x\frac{du}{dx}+(x^{2}-n^{2})u=0.\]
    
%     \item Asympotics
%     \begin{itemize}
%         \item For small values of $x$, that is, as $x \to 0$: $J_{0}(x)\sim 0$, and $\displaystyle J_{n}(x)\sim \frac{1}{2^{n}n!}x^{n}$.
%         \item For large values of $x$, that is, as $x \to \infty $: $\displaystyle J_{n}(x)\sim \sqrt{\frac{2}{\pi x}}\cos\left(x-\frac{\pi}{4}-\frac{n\pi}{2}\right)$.
%     \end{itemize}
    
%     \item \textbf{Theorem 6.1}. Derivative
%     \begin{enumerate}[(i)]
%         \item $\displaystyle \frac{d}{dx}
%         [x^{n} J _{n} (x)] = x ^{n} J _{n-1} (x)$ for $n \geq 1$, and
%         \item $\displaystyle \frac{d}{dx}
%         [x^{-n} J _{n} (x)] = -x ^{-n} J _{n+1} (x)$ for $n \geq 0$.
%     \end{enumerate}

%     \item \textbf{Theorem 6.3}. Integration
%     \[J_{n}(x)=\frac{1}{\pi}\int_{0}^{\pi}\cos(n\theta - x \sin\theta)\, d\theta, \quad n\in \mathbb{Z}.\]

%     \item \textbf{Corollary 6.3.} (Bessel series) For $n \geq 0$ we have
%     \[\sin x = 2\sum_{n=0}^{\infty} (-1)^{n}J_{2n+1}(x), \quad \cos x = J_{0}(x) + 2\sum_{n=1}^{\infty} (-1)^{n}J_{2n}(x).\]

    \item \textbf{Theorem 6.4.} (Orthogonality) For a fixed integer $m \geq 0$,
    \[\int_{0}^{1} x J_{m}(z_{mn}x)J_{m}(z_{mk}x)\,dx=0, \quad \int_{0}^{1} xJ_{m}(z_{mn}x)^{2}\,dx = \frac{1}{2}J_{m+1}(z_{mn})^{2},\]
    where $z_{mn}$ is a zero of $J_{m}(x)$ for $n\geq 1$.
%     That is, the functions $J _{m} (z _{mn} x)$ are orthogonal on the interval $0 \leq x \leq 1$ with respect to the weight function $\sigma(x) = x$.

    \item \textbf{Theorem 6.5.} (Fourier-Bessel Expansion Theorem) If $f$ and $f'$ are piecewise continuous on the interval $0 \leq x \leq 1$, then for $0 < x _{0} < 1$, the \textbf{Fourier-Bessel series} expansion
    \[f(x) = \sum_{n=1}^{\infty} a_{n}J_{m}(z_{mn}x_{0}), \quad \text{where}\quad a_{n} = \frac{2}{J_{m+1}(z_{mn})^{2}} \int _{0}^{1} f(x) J_{m}(z_{mn}x)x\,dx,\]
    converges to $[f(x_{0}^{+})+f({x_{0}^{-}})]/2$. At $x _{0} = 1$, the series converges to $0$, since every $J _{m} (z _{mn} ) = 0$. At $x _{0} = 0$, the series converges to 0 if $m \geq 1$, and to $f (0 ^{+} )$ if $m = 0$.

\end{enumerate}

\newpage

\section{Polar coordinates}

\begin{enumerate}

    \item \textit{Vibrating circular membrane}. Consider the following wave equation in a disk

    \[
    \begin{aligned}
        & \frac{\partial^{2} u}{\partial t^{2}}  = c^{2}\left( \frac{\partial^{2} u}{\partial r^{2}} + \frac{1}{r}\frac{\partial u}{\partial r} + \frac{1}{r^{2}}\frac{\partial^{2} u}{\partial \theta^{2}} \right), \quad 0 < r < a, \quad -\pi < \theta < \pi, \quad t>0, \\
        & u(r, -\pi, t) = u(r, \pi, t), \\
        & \frac{\partial u}{\partial \theta}(r, -\pi,t) = \frac{\partial u}{\partial \theta}(r, \pi,t), \\
        & u(a, \theta, t) = 0, \\
        & |u(r,\theta ,t)|<\infty, \quad \text{as} \quad r\to 0^{+},\\
        & u(r,\theta, 0) = f(r,\theta), \\
        & \frac{\partial u}{\partial t} (r, \theta, 0) = g(r, \theta).
    \end{aligned}
    \]

\newpage

\item \textbf{Exercise 13.14.}

Solve the two-dimensional heat equation inside a disk with circularly symmetric time-independent sources, boundary conditions, and initial conditions:
\[\frac{\partial u}{\partial t} = \frac{k}{r}\frac{\partial}{\partial r}\left( r \frac{\partial u}{\partial r} \right) + Q(r) , \quad 0<r<a, \quad t>0,\]
with
\[u(r,0) = f(r), \quad u(a,t)=T.\]

\newpage 

\textit{(continue Exercise 13.14)}

\newpage 

\textit{(continue Exercise 13.14)}

\newpage 

\textit{(continue Exercise 13.14)}

% \newpage 
% 
% \item \textbf{Exercise 14.18.}
% 
% Solve the wave equation for a ``pie-shaped'' membrane of radius a and angle $\pi/3$ (= 60\textdegree) :
% \[\frac{\partial ^{2}u}{\partial t^{2}} = c^{2}\nabla ^{2}u.\]
% Show that the eigenvalues are all positive. Determine the natural frequencies of oscillation if the boundary conditions are
% \[u(r,0,t) = 0, \quad u \left(r, \frac{\pi}{3}, t \right)=0, \quad \frac{\partial u}{\partial r}(a,\theta, t)=0.\]
% 
% \newpage 
% 
% \textit{(continue Exercise 14.18)}
% 
% \newpage 
% 
% \textit{(continue Exercise 14.18)}


\end{enumerate}





\chapter{Week 11}
\setcounter{weekpage}{1}
\thispagestyle{plainweek}

\section{Spherical Coordinates}

\begin{enumerate}



\item Given a point $P$ with Cartesian coordinates $(x,y,z)$, where $ (x,y) \neq (0,0)$, the \textbf{spherical coordinates} of $P$ are $(r, \theta, \phi)$, where
    \begin{align*}
        x & = r \cos \phi \sin \theta, \\
        y & = r \sin \phi \sin \theta, \\
        z & = r \cos \theta.
    \end{align*}
%     The Jacobian determinant of the transformation is
%     \[\frac{\partial (x,y,z)}{\partial (r,\theta,\phi)} = 
%     \begin{vmatrix}
%         \sin \theta \cos \phi & r \cos \theta \cos \phi & - r \sin \theta \sin \phi \\
%         \sin \theta \sin \phi & r \cos \theta \sin \phi & r \sin \theta \cos \phi \\
%         \cos \theta &  - r \sin \theta & 0 \\
%     \end{vmatrix} = r^{2}\sin\theta.
%     \]


\item The \textbf{Laplacian in spherical coordinates} is
    \[
    \begin{aligned}\nabla^{2} u & = \frac{1}{r^{2}}\frac{\partial}{\partial r}\left( r^{2} \frac{\partial u}{\partial r} \right) + \frac{1}{r^{2}\sin \theta} \frac{\partial}{\partial \theta}\left( \sin\theta \frac{\partial u}{\partial \theta} \right) + \frac{1}{r^{2}\sin^{2}\theta} \frac{\partial^{2} u}{\partial \phi^{2}} \\
    & = \frac{\partial^{2} u}{\partial r^{2}} + \frac{2}{r}\frac{\partial u}{\partial r} + \frac{1}{r^{2}}\left( \frac{\partial^{2} u}{\partial \theta^{2}} + \cot \theta \frac{\partial u}{\partial \theta} + \csc^{2}\theta \frac{\partial^{2}u}{\partial \phi^{2}} \right).
    \end{aligned}
    \]


% \item \textbf{Legendre's equation} is given by
% \[(1-x^{2})\frac{d^{2}v}{dx^{2}}-2x\frac{dv}{dx}+\lambda  v=0, \quad -1<x<1.\]

\item \textbf{Theorem 7.2} The singular Sturm-Liouville problem given by \textbf{Legendre's equation}
\[
\begin{aligned}
& (1-x^{2})\frac{d^{2}v}{dx^{2}}-2x\frac{dv}{dx}+\lambda  v=0, \quad -1<x<1. \\
& |v(x)| \text{  and  } |v'(x)| \text{  bounded as  } x\to -1^{+} \text{  and  } x\to 1^{-}
\end{aligned}
\]
has eigenvalues and corresponding eigenfunctions
\[\lambda_{n}= n(n+1), \quad \phi_{n}(x)=P_{n}(x)=\sum_{k=0}^{\lfloor x \rfloor} \frac{(-1)^{k}(2n-2k)!x^{n-2k}}{2^{n}k!(n-k)!(n-2k)!}, \quad n\geq 0,\]
where $P_{n}(x)$ are called \textbf{Legendre polynomials}.

\item \textbf{Theorem 7.3.} (Orthogonality of Legendre Polynomials) If $m$ and $n$ are nonnegative integers with $m \neq n$,
\[\int_{-1}^{1}P_{m}(x)P_{n}(x)\,dx = 0.\]

\newpage

\item \textbf{Exercise 13.19. Heat Flow on a Spherical Shell}

Consider the flow of heat on a thin conducting spherical shell
\[S = \{ (r,\theta, \phi) \mid r = 1, 0 \leq \theta \leq \pi, −\pi \leq \phi \leq \pi \}. \]
We want to find the temperature distribution $u(\theta, t)$ on the shell if we are given the initial temperature distribution $u(\theta, 0) = f (\theta)$.

\newpage \textit{(continue Exercise 13.19)}


\end{enumerate}

\newpage

\section{Fourier Series}

\begin{enumerate}


\item \textbf{Definition 8.2.} If $f$ is piecewise smooth on every finite interval $(a, b)$ and absolutely integrable on $(-\infty, \infty)$, the \textbf{Fourier transform} of $f(x)$, denoted $\widehat{f}$, is
\[\widehat{f}(\omega) = \mathcal{F} \, [f(x)] \,(\omega) =  \frac{1}{2\pi} \int_{-\infty} ^{\infty}f(x) \,e^{i\omega x} \,dx, \quad -\infty < \omega <\infty.\]


\item \textbf{Theorem 8.4.} If $f$ and $f'$ are piecewise continuous on every finite interval $(a, b)$ and $f$ is absolutely integrable on $(-\infty, \infty)$, then
\[\frac{f(x^{+})+f(x^{-})}{2} = \int_{-\infty}^{\infty}\widehat{f}(\omega)\, e^{-i\omega x}\, d\omega, \quad -\infty <x<\infty.\]

\item \textbf{Definition 8.4.} If $f$ is piecewise smooth on every finite interval $(a, b)$, absolutely integrable on $(-\infty, \infty)$ and $f$ is continuous on $(-\infty, \infty)$, then 
\[{f}(x) = \mathcal{F}^{-1} \, [f(\omega)] \,(x) =  \int_{-\infty} ^{\infty}\widehat{f}(\omega) \,e^{-i\omega x} \,d\omega, \quad -\infty < x <\infty.\]
is called the \textbf{inverse Fourier transform} of $\widehat{f} (\omega)$.

\item \textbf{Properties}
\begin{enumerate}[(i)]
    \item \textbf{Theorem 8.5.} \textit{(Linearity) }
    \begin{enumerate}[(a)]
        \item $\mathcal{F}\,[af+bg] = a\mathcal{F}\,[f] + b \mathcal{F}\,[g]$
        \item $\mathcal{F}^{-1}\,[af+bg] = a\mathcal{F}^{-1}\,[f] + b \mathcal{F}^{-1}\,[g]$
    \end{enumerate}
    \item \textbf{Theorem 8.6.} \textit{(Shift}) 
    \begin{enumerate}[(a)]
        \item $\mathcal{F}\,[f(x-a)](\omega) = e^{ia\omega}\widehat{f}(\omega)$
        \item $\mathcal{F}\,[e^{-iax} f(x-a)](\omega) = \widehat{f}(\omega)$
        \item $\mathcal{F}\,[f(ax)](\omega) = (1/|a|)\widehat{f}(\omega/a)$
    \end{enumerate}
    \item \textbf{Theorem 8.7.} \textit{(Transform of Derivatives) }
    \[\mathcal{F}\,[f^{(n)}(x)] (\omega) = (-i\omega)^{n}\, \mathcal{F}\,[f(x)](\omega)\]
    \item \textbf{Theorem 8.8.} \textit{(Transform of an Integral)}
    \[\mathcal{F}\left[ \int_{0}^{x}f(s)\,ds \right] (\omega) = -\frac{1}{i\omega} \mathcal{F}\, [f(x)](\omega)\]
\end{enumerate}


% \newpage
% 
% 
% \item Proof the linearity, shift, derivative and integral properties.



\newpage


\item \textbf{Definition 8.5.} Let $f : [0, \infty) \to \mathbb{R}$ be continuous and absolutely integrable on $(0, \infty)$, and let $f '$ be piecewise continuous on every finite interval $(a, b) \subset
(0, \infty)$. Then the sine and cosine transform and inverse transform are given by:
\begin{enumerate}[(i)]
    \item The \textbf{Fourier sine transform of} $f (x)$ and the \textbf{inverse sine transform of} $g(\omega)$ are
    \[\mathcal{S}\,[f(x)](\omega)=\frac{2}{\pi} \int_{0}^{\infty}f(x)\sin\omega x \, dx, \quad \mathcal{S}^{-1}\,[g(\omega)](x)= \int_{0}^{\infty}g(\omega)\sin\omega x \, d\omega,\]
    \item The \textbf{Fourier cosine transform of} $f (x)$ and the \textbf{inverse cosine transform of} $g(\omega)$ are
    \[\mathcal{C}\,[f(x)](\omega)=\frac{2}{\pi} \int_{0}^{\infty}f(x)\cos\omega x \, dx, \quad \mathcal{C}^{-1}\,[g(\omega)](x)= \int_{0}^{\infty}g(\omega)\cos\omega x \, d\omega,\]
\end{enumerate}


\item \textbf{Theorem 8.10.} \textit{(Sine and Cosine Transforms of Derivatives)}

If $f$ is piecewise smooth, $f$ and $f ' $ are integrable on $[0, \infty)$, and $\lim _{x\to \infty} f (x) \to 0$, then:
\begin{enumerate}[(a)]
    \item For the Fourier sine transform, we have
    \[\mathcal{S}\,[f'(x)]\,(\omega) = -\omega\, \mathcal{C}\,[f(x)]\,(\omega)\]
    and if $f ''$ is integrable on $[0, \infty)$ and $\lim _{x\to \infty} f' (x) \to 0$ also, then
    \[\mathcal{S}\,[f''(x)]\,(\omega) = \frac{2\omega}{\pi}f(0) -\omega^{2} \,\mathcal{S}\,[f(x)]\,(\omega).\]
    \item For the Fourier cosine transform, we have
    \[\mathcal{C}\,[f'(x)]\,(\omega) = -\frac{2}{\pi} f(0) + \omega\, \mathcal{S}\,[f(x)]\,(\omega)\]
    and if $f ''$ is integrable on $[0, \infty)$ and $\lim _{x\to \infty} f' (x) \to 0$ also, then
    \[\mathcal{C}\,[f''(x)]\,(\omega) = -\frac{2}{\pi}f'(0) -\omega^{2} \,\mathcal{C}\,[f(x)]\,(\omega).\]
\end{enumerate}




\newpage


\item \textbf{Definition 8.6.} (Convolution Product)

If $f$ and $g$ are defined on all of $\mathbb{R}$, and are integrable over $\mathbb{R}$, the \textbf{convolution of} $f$ \textbf{and} $g$, denoted $f*g$, is given by
\[(f*g)(x) = \int _{-\infty}^{\infty} f(x-t)g(t)\, dt,\quad -\infty <x<\infty.\]




\item \textit{Example 8.6. (Convolution with a Sine)}

Let $f$ be an even integrable function on $\mathbb{R}$, and let $g(x) = \sin ax$ for $x \in \mathbb{R}$, where $a > 0$ is constant; then
\[ (f * g) (x) = 2\pi\sin (a x) \, \widehat f (a),\]
where $\hat f$ is the Fourier transform of $f$.

\newpage 

\item \textbf{Theorem 8.11.} \textit{(Convolution Theorem)}

If $f$ and $g$ are integrable and satisfy the hypotheses of Theorem 8.4, then
\begin{enumerate}[(a)]
    \item $\displaystyle \mathcal[F]\, \left[ \frac{1}{2\pi} f * g\right] = \widehat{f}\cdot \widehat{g}.$
    \item If, in addition, $f$ and $g$ are continuous, then $f * g = 2\pi \, \mathcal{F} ^{-1} \left[  \widehat{f} \cdot \widehat{g} \right] $.
\end{enumerate}


\item \textbf{Theorem 8.12.} If the function $f : \mathbb{R} \to \mathbb{R}$ is piecewise smooth on every
finite interval and is absolutely integrable on $\mathbb{R}$, then the Fourier transform $\widehat{f}(\omega)$ is uniformly continuous on $\mathbb{R}$.


\item \textit{Example 8.7.} Find the Fourier transform of the function
\[ g(x)=
\begin{cases}
    \displaystyle 1 - \frac{|x|}{2}, & \text{for } |x|<2,\\
    0, & \text{for } |x|\geq 2.
\end{cases}\]

    \newpage 
    \item \textit{Example 8.8.}
    Let $f(x)$ be the rectangular pulse
    \[ f(x)=
    \begin{cases}
        1, & \text{for } |x| < 1,\\
        0, & \text{for } |x| > 1.
    \end{cases}\]
    and $f (-1) = f (1) = \frac{1}{2}$. Let $h(x)$ be the convolution of $f$ with itself, that is,
    \[h(x)=\int_{-\infty}^{\infty} f(x-t) f(t) dt.\]
    Find the Fourier transform of $h(x)$, and use the convolution theorem to identify $h(x)$.


\end{enumerate}


\chapter{Week 12}
\setcounter{weekpage}{1}
\thispagestyle{plainweek}

\section{Fourier Transform Methods in PDEs}

\begin{enumerate}


\item For the heat equation and wave equation, we define
\[\widehat{u}(\omega,t) = \mathcal{F}\left[ u(x,t) \right](\omega) = \frac{1}{2\pi}\int_{-\infty}^{\infty} u(x,t) e^{i\omega x} \, dx. \]
and recall the following operational properties of the Fourier transform:
\begin{enumerate}[i)]
    \item $\displaystyle \mathcal{F}\left[ \frac{\partial^{n} u}{\partial t^{n}}(x,t) \right](\omega) = \frac{d^{n}}{dt^{n}}\widehat{u}(\omega, t)$.
    \item $\displaystyle \mathcal{F}\left[ \frac{\partial^{n} u}{\partial x^{n}}(x,t) \right](\omega) = (-i\omega)^{n}\widehat{u}(\omega, t)$.
\end{enumerate}


\item \textit{Example 9.1}. Consider the wave problem
\begin{align*}
    & \frac{\partial ^{2} u}{\partial t^{2}}=25 \frac{\partial ^{2} u}{\partial x^{2}}, \quad -\infty<x<\infty, \quad t>0,\\
    & u(x,0)=f(x)=\begin{cases}
              1, & x>0,\\
              0, & x<0,
             \end{cases} \\
    & \frac{\partial u}{\partial t}(x,0)= 0.
\end{align*}

\newpage

\item \textbf{Theorem 9.1.} The solution $u(x, t)$ of the linear heat equation
\begin{align*}
    & \frac{\partial u}{\partial t}=k \frac{\partial ^{2} u}{\partial x^{2}}, \quad -\infty<x<\infty, \quad t>0,\\
    & u(x,0)=f(x), \\
    & |u(x,t)| \quad \text{bounded as} \quad x\to \infty
\end{align*}
can be written as
\[u(x,t)=f(x)*G(x,t)=\int_{-\infty}^{\infty}f(\xi)G(\xi-x,t)\,d\xi,\]
where
\[G(x,t) = \frac{1}{\sqrt{4\pi k t}} e^{-\frac{x^{2}}{4kt}}\]
is called the \textbf{fundamental solution} of the heat equation.


\item The \textbf{error function} is given by
\[\text{erf}(x)=\frac{2}{\sqrt{\pi}} \int_{0}^{x}s^{-t^{2}}\, dt. \]


\item \textbf{Lemma 9.1.} The error function is a monotone increasing function which satisfies
\[\lim_{x\to -\infty} \text{erf}(x)=-1,\quad \text{and}\quad \lim_{x\to -\infty} \text{erf}(x)=1.\]




\item \textbf{Exercise 16.12}
Use Fourier transforms to find the solution to
\begin{align*}
    & \frac{\partial u}{\partial t}= \frac{\partial ^{2} u}{\partial x^{2}}, \quad -\infty<x<\infty, \quad t>0,\\
    & u(x,0)=\begin{cases}
               100, & |x|<1 \\
               0, & |x|>1.
             \end{cases}
\end{align*}
in terms of the error function.

\newpage \textit{(continue Exercise 16.12.)}

\newpage

\item \textbf{Heat Flow in a Semi-infinite Rod} 

\begin{enumerate}[(i)]
 \item  The heat equation on a semi-infinite domain with \textbf{Dirichlet} condition
\begin{align*}
    & \frac{\partial u}{\partial t}=k \frac{\partial ^{2} u}{\partial x^{2}}, \quad 0<x<\infty, \quad t>0,\\
    & u(0,t) = 0, \\
    & u(x,0)=f(x), \\
    & |u(x,t)| \quad \text{bounded as} \quad x\to \infty
\end{align*}
has solution
\[u(x,t) = f_{\text{odd}}(x)*G(x,t) = \frac{1}{\sqrt{4k\pi t}}\int_{0}^{\infty} f(s) \left( e^{-(x-s)^{2}/4kt} - e^{-(x+s)^{2}/4kt} \right) ds. \]


\item The heat equation on a semi-infinite domain with \textbf{Neumann} condition
\begin{align*}
    & \frac{\partial u}{\partial t}=k \frac{\partial ^{2} u}{\partial x^{2}}, \quad 0<x<\infty, \quad t>0,\\
    & \frac{\partial u}{\partial t}(0,t) = 0, \\
    & u(x,0)=f(x), \\
    & |u(x,t)| \quad \text{bounded as} \quad x\to  \infty
\end{align*}
has solution
\[u(x,t) = f_{\text{even}}(x)*G(x,t) = \frac{1}{\sqrt{4k\pi t}}\int_{0}^{\infty} f(s) \left( e^{-(x-s)^{2}/4kt} + e^{-(x+s)^{2}/4kt} \right) ds. \]

\end{enumerate}




\end{enumerate}





\chapter{Week 13}
\setcounter{weekpage}{1}
\thispagestyle{plainweek}

\section{Summary}

\begin{enumerate}


\item \textbf{Separations of variables}
\begin{enumerate}[1)]
\item Write $u(x, t) = X(x)T (t)$.
\item Solve the Sturm-Liouville problem for $X(x)$.
\item Solve the corresponding time problem for $T (t)$.
\item Use superposition.
\item Use the initial conditions.
\end{enumerate}

\item \textbf{Definition 2.10}.  The \textbf{Fourier series}  of $f$ on $(a, b)$ is given by
\[f(x) \sim a_{0}+\sum_{n=1}^{\infty}a_{n}\cos \frac{n\pi x}{l} + b_{n}\sin \frac{n\pi x}{l},\]
where $l=(b-a)/2$ and
\[a_{0} = \frac{1}{2l}\int_{a}^{b}f(x)dx, \quad a_{n} = \frac{1}{l}\int_{a}^{b}f(x)\cos\frac{n\pi x}{l}dx,\quad b_{n} = \frac{1}{l}\int_{a}^{b}f(x)\sin\frac{n\pi x}{l}dx,\quad n\geq 1,\]
are called the \textbf{Fourier coefficients} of $f$.

% \item Let function $f$ defined on $(0,l)$.
% \begin{itemize}
%     \item[(i)] The \textbf{Fourier sine series} for $f$ is
%     \[f(x) \sim \sum_{n=1}^{\infty}b_{n}\sin\frac{n\pi x}{l},\]
%     where
%     \[b_{n} = \frac{2}{l}\int_{0}^{l}f(x)\sin\frac{n\pi x}{l}dx \quad \text{for} \quad n \geq 1. \]
%     Note that this defines $f_{\text{even}}$, the  odd extension of $f$ on $(-l,l)$.
%     \item[(ii)] The \textbf{Fourier cosine series} for $f$ is
%     \[f(x) \sim a_{0} + \sum_{n=1}^{\infty}a_{n}\sin\frac{n\pi x}{l},\]
%     where
%     \[a_{0} = \frac{1}{l}\int_{0}^{l}f(x)dx, \quad \text{and} \quad a_{n} = \frac{2}{l}\int_{0}^{l}f(x)\cos\frac{n\pi x}{l}dx\quad \text{for} \quad n \geq 1. \]
%     Note that this defines $f_{\text{even}}$, the even extension of $f$ on $(-l,l)$.
% \end{itemize}



% \item Standard homogeneous Heat and Wave equations
% 
% \end{enumerate}
% 
% 
% \begin{minipage}{\linewidth}
%     \begin{minipage}{0.5\linewidth}
%         \textbf{Heat eq. with Dirichlet BCs}
%         \[
%         \begin{aligned}
%             & u_{t} = k  u_{xx}, \quad 0<x<a , \quad t>0, \\
%             & u(0,t)=0, \quad t>0,\\
%             & u(a,t)=0, \quad t>0, \\
%             & u(x,0)=f(x), \quad 0<x<a.
%         \end{aligned}
%         \]
%         The solution has the form
%         \[u(x,t)=\sum_{n=1}^{\infty} b_{n}e^{-\left(\frac{n\pi}{a}\right)^{2}kt}\sin \frac{n\pi x}{a}.\]
%     \end{minipage}
%     \begin{minipage}{0.5\linewidth}
%         \textbf{Heat equation with Neumann BCs}
%         \[
%         \begin{aligned}
%             & u_{t} = k  u_{xx}, \quad 0<x<a , \quad t>0, \\
%             & u_{x}(0,t)=0, \quad t>0,\\
%             & u_{x}(a,t)=0, \quad t>0, \\
%             & u(x,0)=f(x), \quad 0<x<a.
%         \end{aligned}
%         \]
%         The solution has the form
%         \[u(x,t)=a_{0}+\sum_{n=1}^{\infty} a_{n}e^{-\left(\frac{n\pi}{a}\right)^{2}kt}\cos \frac{n\pi x}{a}.\]
%     \end{minipage} \\
%     
%     \vspace{20pt}
%     \begin{minipage}{0.5\linewidth}
%         \textbf{Wave equation with Dirichlet BCs}
%         \[
%         \begin{aligned}
%             & u_{tt} = c^{2}  u_{xx}, \quad 0<x<a , \quad t>0, \\
%             & u(0,t)=0, \quad t>0,\\
%             & u(a,t)=0, \quad t>0, \\
%             & u(x,0)=f(x), \quad 0<x<a, \\
%             & u_{t}(x,0)=g(x), \quad 0<x<a.
%         \end{aligned}
%         \]
%         The solution has the form
%         \begin{multline*}
%             u(x,t)=\\ 
%             \sum_{n=1}^{\infty} \left(a_{n}\cos \frac{n\pi c t}{a} + b_{n}\sin \frac{n\pi c t}{a} \right) \sin \frac{n\pi x}{a}.
%         \end{multline*}
%     \end{minipage}
%     \begin{minipage}{0.5\linewidth}
%         \textbf{Wave equation with Neumann BCs}
%         \[
%         \begin{aligned}
%             & u_{tt} = c^{2}  u_{xx}, \quad 0<x<a , \quad t>0, \\
%             & u_{x}(0,t)=0, \quad t>0,\\
%             & u_{x}(a,t)=0, \quad t>0, \\
%             & u(x,0)=f(x), \quad 0<x<a, \\
%             & u_{t}(x,0)=g(x), \quad 0<x<a.
%         \end{aligned}
%         \]
%         The solution has the form
%         \begin{multline*}
%             u(x,t)= a_{0} + \\ 
%             \sum_{n=1}^{\infty} \left(a_{n}\cos \frac{n\pi c t}{a} + b_{n}\sin \frac{n\pi c t}{a} \right) \cos \frac{n\pi x}{a}.
%         \end{multline*}
%     \end{minipage}
% \end{minipage}
% 
% 
% \begin{enumerate}
% 
% \setcounter{enumi}{3}

\item \textbf{Method of Characteristics}

Consider the first-order linear time-dependent problem of the form
\[
\begin{aligned}
    & \frac{\partial u}{\partial t} + B(x,t)\frac{\partial u}{\partial x} = C(x,t,u) , \quad -\infty<x<\infty, \quad t>0, \\
    & u(x,0)=f(x).
\end{aligned}
\]
The method of characteristic consists on solving the \textbf{characteristic equations}
\[
\begin{aligned}
    & \frac{d x}{d t} = B(x,t),\\
    & \frac{d u}{d t} = C(x,t,u),
\end{aligned}
\]
and then using the initial condition.


\item Consider the one-dimensional wave equation
\[\frac{\partial^{2} u}{\partial t^{2}} = c^{2} \frac{\partial^{2} u}{\partial x^{2}}, \quad -\infty<x<\infty, \quad t>0, \quad u(x,0)=f(x), \quad \frac{\partial u}{\partial t}(x,0)=g(x).\]
\textbf{d'Alembert's solution} is given by
\[u(x,t)= \frac{1}{2}[f(x+ct)+f(x-ct)] + \frac{1}{2c} \int _{x-ct}^{x+ct} g(\mu)\, d\mu.\]

\item Consider the one-dimensional wave equation
\[
\begin{aligned}
& \frac{\partial^{2} u}{\partial t^{2}} = c^{2} \frac{\partial^{2} u}{\partial x^{2}}, \quad 0<x<l, \quad t>0, \\
& u(0,t)=0, \quad  u(l,t)=0, \quad u(x,0)=f(x), \quad \frac{\partial u}{\partial t}(x,0)=g(x).\end{aligned}
\]
\textbf{d'Alembert's solution} is given by
\[u(x,t)= \frac{1}{2}[\bar f_{\text{odd}}(x+ct)+ \bar f_{\text{odd}}(x-ct)] + \frac{1}{2c}\int _{x-ct}^{x+ct} \bar g_{\text{odd}}(\mu)\, d\mu,\]
where $\bar f_{\text{odd}}$ and $\bar g _{\text{odd}}$ are the $2l$-periodic extension of $f$ and $g$, respectively.



\item A regular \textbf{Sturm-Liouville problem},
\begin{align*}
& (p(x)\phi ' ) ' + [q(x) + \lambda \sigma(x)]\phi = 0, \quad a < x < b,\\
& \alpha _1 \phi(a) + \beta _1 \phi ' (a) = 0, \\
& \alpha _2 \phi(b) + \beta _2 \phi ' (b) = 0,
\end{align*}
has eigenvalues $\lambda_{n}$ and corresponding eigenfunctions $\phi_{n}$ for $n\geq 1$.

\item \textbf{Theorem 4.6.} \textit{(Dirichlet's Theorem)}

If $f$ is piecewise smooth on $[a, b]$, the \textbf{generalized Fourier series},
\[
f(x)\sim \sum_{n=1}^{\infty} c_{n}\phi_{n}(x), \quad 
\text{where}, \quad
c_{n}=\frac{\langle f, \phi_{n}\rangle}{\|\phi_{n}\|^{2}} = \frac{1}{\|\phi_{n}\|^{2}}\int_{a}^{b} f(x)\phi_{n}(x)\sigma(x)\,dx,
\]
for $n \geq 1$, converges pointwise to $[f (x ^{+} ) + f (x ^{-} )]/2$ for each $x \in (a, b)$.



\item \textbf{Method of eigenfunction expansions} 
\begin{enumerate}[1)]
    \item Identify the eigenfunctions $\phi_{n}$ associated to the problem.
    \item Assume a solution of the form $u(x,t) = \sum_{n=1}^{\infty} a_{n}(t)\phi_{n}(x)$. 
    \item Expand initial conditions and other related functions using generalized Fourier series.
    \item Substitute into the equation to solve for $a_{n}(t)$.
\end{enumerate}


\item \textbf{Theorem 4.7.} 

$\lambda _{n}$ can be calculated from the \textbf{Rayleigh quotient}:
\[
\lambda_{n} = \frac{ \displaystyle -p(x)\phi_{n}(x)\phi'_{n}(x)\Big|_{a}^{b} + \int_{a}^{b}(p(x)\phi'_{n}(x)^{2} - q(x)\phi_{n}(x)^{2}) \, dx }{ \displaystyle \int_{a}^{b}\phi_{n}(x)^{2}\sigma(x)\, dx}.
\]

\newpage
\item Summary of Sturm-Liouville problems.

\end{enumerate}

\begin{center}\small
\begin{tabular}{|c|c|c|c|}
    \hline
    \parbox[c][30pt]{90pt}{\centering \textbf{Model Type}} & 
    \parbox[c][30pt]{90pt}{\centering \textbf{S-L Problem}} & \parbox[t]{90pt}{\centering \textbf{Spectrum}} & 
    \parbox[c][30pt]{90pt}{\centering \textbf{Eigenfunctions}} \\
    \hline
    
    \parbox[c][30pt]{90pt}{\centering \textbf{Homogeneous \\ \vspace{10pt} Dirichlet B.C.}} & 
    \parbox[c][60pt]{90pt}{\centering $\phi''(x) + \lambda \phi(x)=0$ \\ \vspace{10pt} $\phi(0)=\phi(l)=0$}
    & 
    \parbox[c][60pt]{90pt}{\centering $\displaystyle \lambda_{n}=\left( \frac{n\pi}{l} \right)^{2}$ \\ \vspace{10pt} $n=1, 2, \cdots$} & 
    \parbox[c][60pt]{90pt}{\centering $\displaystyle \phi_{n} = \sin\frac{n\pi x}{l}$ \\ \vspace{10pt} $n=1, 2, \cdots$} \\
    \hline 
    
    \parbox[c][30pt]{90pt}{\centering \textbf{Homogeneous \\ \vspace{10pt} Neumann B.C.}} & 
    \parbox[c][60pt]{90pt}{\centering $\phi''(x) + \lambda \phi(x)=0$ \\ \vspace{10pt} $\phi'(0)=\phi'(l)=0$}
    & 
    \parbox[c][60pt]{90pt}{\centering $\displaystyle \lambda_{n}=\left( \frac{n\pi}{l} \right)^{2}$ \\ \vspace{10pt} $n=0, 1, \cdots$} & 
    \parbox[c][60pt]{90pt}{\centering $\displaystyle \phi_{n} = \cos\frac{n\pi x}{l}$ \\ \vspace{10pt} $n=0, 1, \cdots$} \\
    \hline
    
    \parbox[c][30pt]{90pt}{\centering \textbf{Mixed \\ \vspace{10pt} Type I}} & 
    \parbox[c][60pt]{90pt}{\centering $\phi''(x) + \lambda \phi(x)=0$ \\ \vspace{10pt} $\phi(0)=\phi'(l)=0$}
    & 
    \parbox[c][60pt]{100pt}{\centering $\displaystyle \lambda_{n}=\left( \frac{(2n-1)\pi}{2l} \right)^{2}$ \\ \vspace{10pt} $n=1, 2, \cdots$} & 
    \parbox[c][60pt]{100pt}{\centering $\displaystyle \phi_{n} = \sin\frac{(2n-1)\pi x}{2l}$ \\ \vspace{10pt} $n=1, 2, \cdots$} \\
    \hline
    
    \parbox[c][30pt]{90pt}{\centering \textbf{Mixed \\ \vspace{10pt} Type II}} & 
    \parbox[c][60pt]{90pt}{\centering $\phi''(x) + \lambda \phi(x)=0$ \\ \vspace{10pt} $\phi'(0)=\phi(l)=0$}
    & 
    \parbox[c][60pt]{100pt}{\centering $\displaystyle \lambda_{n}=\left( \frac{(2n-1)\pi}{2l} \right)^{2}$ \\ \vspace{10pt} $n=1, 2, \cdots$} & 
    \parbox[c][60pt]{100pt}{\centering $\displaystyle \phi_{n} = \cos\frac{(2n-1)\pi x}{2l}$ \\ \vspace{10pt} $n=1, 2, \cdots$} \\
    \hline
    
    \parbox[c][30pt]{90pt}{\centering \textbf{Periodicity \\ \vspace{10pt} conditions}} & 
    \parbox[c][80pt]{90pt}{\centering $\phi''(\theta) + \lambda \phi(\theta)=0$ \\ \vspace{10pt} $\phi(-\pi)=\phi(\pi)$ \\ \vspace{10pt} $\phi'(-\pi)=\phi'(\pi)$}
    & 
    \parbox[c][60pt]{100pt}{\centering $\displaystyle \lambda_{n}=n^{2}$ \\ \vspace{10pt} $n=0, 1, \cdots$} & 
    \parbox[c][60pt]{120pt}{\centering $ \phi_{n} = a_{n}\cos n \theta + b_{n}\sin n \theta$ \\ \vspace{10pt} $n=0, 1, \cdots$} \\
    \hline
    
    \parbox[c][30pt]{90pt}{\centering \textbf{Bessel \\ \vspace{10pt} equation}} & 
    \parbox[c][80pt]{130pt}{\centering $r^{2}u'' + r u'+ (\lambda r^{2} - m^{2}) u=0$ \\ \vspace{10pt} $u(a)=0$ and $|u(r)|$ \\ \vspace{10pt} bounded as $r\to 0^{+}$}
    & 
    \parbox[c][60pt]{100pt}{\centering $\displaystyle \lambda_{mn}=\left( \frac{z_{mn}}{a} \right)^{2}$ \\ \vspace{10pt} $n=1, 2, \cdots$} & 
    \parbox[c][60pt]{120pt}{\centering $ u_{mn} = a_{mn}J_{m}\left( \frac{z_{mn}r}{a} \right)$ \\ \vspace{10pt} $n=1, 2, \cdots$} \\
    \hline
    
    \parbox[c][30pt]{90pt}{\centering \textbf{Legendre \\ \vspace{10pt} equation}} & 
    \parbox[c][80pt]{120pt}{\centering $(\sin\theta \,v')' + \lambda \sin \theta v=0$ \\ \vspace{10pt} $|v(\theta)|$ and $|v'(\theta)|$ \\ \vspace{10pt} bounded as $\theta \to \pm 1$}
    & 
    \parbox[c][60pt]{100pt}{\centering $\displaystyle \lambda_{n}=n(n+1)$ \\ \vspace{10pt} $n=0, 1, \cdots$} & 
    \parbox[c][60pt]{120pt}{\centering $ v_{n} = a_{n}P_{ n} (\cos \theta )$\\ \vspace{10pt} $n=0, 1, \cdots$} \\
    \hline
\end{tabular}
\end{center}


\begin{enumerate}

\setcounter{enumi}{9}

\item The \textbf{Laplacian in polar coordinates} is
\[\nabla^{2} u = \frac{1}{r}\frac{\partial}{\partial r}\left( r\frac{\partial u}{\partial r} \right) +\frac{1}{r^{2}}\frac{\partial^{2} u}{\partial \theta^{2}} = \frac{\partial^{2} u}{\partial r^{2}} + \frac{1}{r}\frac{\partial u}{\partial r} + \frac{1}{r^{2}}\frac{\partial^{2} u}{\partial \theta^{2}}.\]

\item The \textbf{Laplacian in spherical coordinates} is
\[
\begin{aligned}\nabla^{2} u & = \frac{1}{r^{2}}\frac{\partial}{\partial r}\left( r^{2} \frac{\partial u}{\partial r} \right) + \frac{1}{r^{2}\sin \theta} \frac{\partial}{\partial \theta}\left( \sin\theta \frac{\partial u}{\partial \theta} \right) + \frac{1}{r^{2}\sin^{2}\theta} \frac{\partial^{2} u}{\partial \phi^{2}} \\
& = \frac{\partial^{2} u}{\partial r^{2}} + \frac{2}{r}\frac{\partial u}{\partial r} + \frac{1}{r^{2}}\left( \frac{\partial^{2} u}{\partial \theta^{2}} + \cot \theta \frac{\partial u}{\partial \theta} + \csc^{2}\theta \frac{\partial^{2}u}{\partial \phi^{2}} \right).
\end{aligned}
\]


\item \textbf{Fourier Integral Representation} of $f$ on $(-\infty,\infty)$

\[f(x) = \int _{0}^{\infty} \left[A(\omega)\cos \omega x + B(\omega) \sin \omega x \right]\, d\omega\]
where 
\[A(\omega) = \frac{1}{\pi}\int _{-\infty}^{\infty} f(x)\cos \omega x \, dx ,\quad B(\omega) = \frac{1}{\pi} \int_{-\infty}^{\infty} f(x) \sin \omega x \,dx.\]

\item \textbf{Fourier Cosine Integral Representation} of $f$ on $[0,\infty)$
\[f(x) = \int _{0}^{\infty} A(\omega)\cos \omega x \, d\omega, \quad \text{where} \quad A(\omega) = \frac{2}{\pi}\int _{0}^{\infty} f(x)\cos \omega x \, dx .\]


\item \textbf{Fourier Sine Integral Representation} of $f$ on $[0,\infty)$
\[f(x) = \int _{0}^{\infty} B(\omega)\sin \omega x \, d\omega, \quad \text{where} \quad B(\omega) = \frac{2}{\pi}\int _{0}^{\infty} f(x)\sin \omega x \, dx .\]


\item \textbf{Fourier transform} of $f(x)$ on $(-\infty,\infty)$,
\[\widehat{f}(\omega) = \mathcal{F} \, [f(x)] \,(\omega) =  \frac{1}{2\pi} \int_{-\infty} ^{\infty}f(x) \,e^{i\omega x} \,dx, \quad -\infty < \omega <\infty.\]
\[{f}(x) = \mathcal{F}^{-1} \, [\widehat{f}(\omega)] \,(x) =  \int_{-\infty} ^{\infty}\widehat{f}(\omega) \,e^{-i\omega x} \,d\omega, \quad -\infty < x <\infty.\]

\item \textbf{Fourier cosine transform} of $f (x)$ on $[0,\infty)$
\[\mathcal{C}\,[f(x)](\omega)=\frac{2}{\pi} \int_{0}^{\infty}f(x)\cos\omega x \, dx, \quad \mathcal{C}^{-1}\,[g(\omega)](x)= \int_{0}^{\infty}g(\omega)\cos\omega x \, d\omega.\]

\item \textbf{Fourier sine transform} of $f (x)$ on $[0,\infty)$
\[\mathcal{S}\,[f(x)](\omega)=\frac{2}{\pi} \int_{0}^{\infty}f(x)\sin\omega x \, dx, \quad \mathcal{S}^{-1}\,[g(\omega)](x)= \int_{0}^{\infty}g(\omega)\sin\omega x \, d\omega.\]

\item \textbf{Gauss kernel}
\[g(x,t) = \frac{1}{\sqrt{4\pi k t}}e^{-\frac{x^{2}}{4 k t}}.\]

\item \textbf{Error function}
\[\text{erf}(x):= \frac{2}{\sqrt{\pi}}\int_{0}^{x}e^{-t^{2}}\,dt.\]

\end{enumerate}

\newpage

\section{Final review}

\begin{enumerate}

\item \textit{Example 3.4 modified.} Consider the following nonhomogeneous one-dimensional heat equation:

\[
\begin{aligned}
    & \frac{\partial u}{\partial t}  = k \frac{\partial^{2} u}{\partial x^{2}} + f(x,t) , \quad 0 < x < 2, \quad t>0, \\
    & u(0, t) = 0, \\
    & \frac{\partial u}{\partial x}(2, t) = 0, \\
    & u (x, 0) = g(x),
\end{aligned}
\]
where $f(x, t)$ is a continuous function of $x$ and $t$.

\newpage

\textit{(continue Example 3.4.)}

\newpage

\item \textit{Example 3.4 modified part 2.} Consider the following nonhomogeneous one-dimensional heat equation:

\[
\begin{aligned}
    & \frac{\partial u}{\partial t}  = k \frac{\partial^{2} u}{\partial x^{2}} + f(x,t) , \quad 0 < x < 2, \quad t>0, \\
    & u(0, t) = a(t), \\
    & \frac{\partial u}{\partial x}(2, t) = b(t), \\
    & u (x, 0) = g(x)
\end{aligned}
\]
where $f(x, t)$ is a continuous function of $x$ and $t$, and $a(t)$ and $b(t)$ are
continuously differentiable functions of $t$.


\newpage

\item \textbf{Exercise 19.9.} Consider torsional oscillations of a homogeneous cylindrical shaft. If $\omega(x, t)$ is the angular displacement at time $t$ of the cross section at $x$, then
\[\frac{\partial^{2} \omega}{\partial t^{2}} = a^{2} \frac{\partial^{2} \omega}{\partial x^{2}}, \quad 0<x<l,\quad t>0, \]
where the initial conditions are
\[\omega(x,0)=f(x) \quad \text{and} \quad \frac{\partial \omega}{\partial t} (x,0)=0,\]
and the ends of the shaft are fixed elastically:

\[\frac{\partial \omega}{\partial x} (0,t) -\alpha\, \omega(0,t)=0 \quad \text{and} \quad \frac{\partial \omega}{\partial x} (l,t) + \alpha\, \omega(l,t)=0 \]
with $\alpha$ a positive constant.

\begin{enumerate}[(a)]
\item Why is it possible to use separation of variables to solve this problem?

\item Use separation of variables and show that one of the resulting problems is a regular Sturm-Liouville problem.

\item Show that all of the eigenvalues of this regular Sturm-Liouville problem are positive.

\end{enumerate}

\textbf{Note:} You do not need to solve the initial value problem, just answer the questions (a), (b), and
(c).

\newpage

\textit{(continue Exercise 19.19.)}

\newpage


\item How to solve PDE's in practice?

\end{enumerate}









\end{document}




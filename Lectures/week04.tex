\chapter{Week 4}
\setcounter{weekpage}{1}
\thispagestyle{plainweek}

\section[Separation of Variables]{Separation of Variables: Homogeneous equations}

\begin{enumerate}



\item In the method of \textbf{separations of variables} we look for solutions of the form
\[u(x,t)=X(x)T(t).\]


\item The \textbf{eigenvalue problem with homogeneous Dirichlet boundary conditions}
\[X''+\lambda X = 0 \, \quad X(0)=0, \quad X(l)=0,\]
has nontrivial solution for eigenvalues and corresponding eigenfunctions
\[\lambda_{n} = \left(\frac{n\pi}{l}\right)^{2}, \quad  X_{n}(x)= \sin \frac{n\pi x}{l}, \quad n\geq 1.\]


\item The \textbf{eigenvalue problem with homogeneous Neumann boundary conditions}
\[X''+\lambda X = 0 \, \quad X'(0)=0, \quad X'(l)=0,\]
has nontrivial solution for eigenvalues and corresponding eigenfunctions
\[\lambda_{n} = \left(\frac{n\pi}{l}\right)^{2}, \quad  X_{n}(x)= \cos \frac{n\pi x}{l}, \quad n\geq 0.\]


\newpage

\item \textbf{Exercise 13.2}

Solve the homogeneous Dirichlet problem for the heat equation
\[\frac{\partial u}{\partial t} = k \frac{\partial^{2}u}{\partial x^{2}}, \quad 0< x < a, \quad t>0,\]
subject to the boundary conditions
\[u(0,t)=0, \quad \text{and} \quad u(a,t)=0,\]
for $t > 0$, with initial conditions
\[u(x,0)=
\begin{cases}
    1, & 0<x<\frac{a}{2} \\
    2, & \frac{a}{2} \leq x< a.
\end{cases}
\]


\newpage


\item \textbf{Exercise 13.3}

Solve the following boundary value--initial value problem for the heat equation
\[\frac{\partial u}{\partial t} = k \frac{\partial^{2}u}{\partial x^{2}},\]
\[u(0,t)=u(a,t)=0,\]
\[u(x,0)=3 \sin \frac{\pi x}{a} - \sin \frac{3\pi x}{a}
\]
for $0<x<a$, $t > 0$.


\newpage

\item \textbf{Exercise 14.2}

Solve the following boundary value--initial value problem for the wave equation:

\[
\begin{aligned}
    & \frac{\partial ^{2}u}{\partial t^{2}} =  \frac{\partial^{2}u}{\partial x^{2}}, \quad 0<x<1, \quad t>0 \\
    & u(0,t)=0, \\
    & u(1,t)=0, \\
    & u(x,0)=\sin \pi x + \tfrac{1}{2}\sin 3\pi x + 3 \sin 7 \pi x, \\
    & \frac{\partial u}{\partial t}(x,0)= 1.
\end{aligned}
\]

You can use the fact the $\left\{ \sin\frac{(2n+1)x}{2} \right\}_{n\geq0}$ are orthogonal on $[0,\pi]$.

\newpage 

\textit{(continue Exercise 14.2)}


\newpage

\item \textbf{Exercise 13.8}

Solve the problem of heat transfer in a bar of length $a = \pi$ and thermal diffusivity $k = 1$, with initial heat distribution $u(x, 0) = \sin x$, where one end of the bar is kept at a constant temperature $u(0, t) = 0$, while there is no heat loss at the other end of the bar, so that $\partial u(\pi, t)/\partial x = 0$, that
is, solve the boundary value--initial value problem
\[
\begin{aligned}
    & \frac{\partial u}{\partial t} = k  \frac{\partial^{2}u}{\partial x^{2}}, \quad 0<x<\pi, \quad t>0 \\
    & u(0,t)=0, \\
    & \frac{\partial u}{\partial x}u(\pi,t)=0, \\
    & u(x,0)=\sin x.
\end{aligned}
\]

\newpage 

\textit{(continue Exercise 13.8)}

\newpage 

\textit{(continue Exercise 13.8)}

\end{enumerate}





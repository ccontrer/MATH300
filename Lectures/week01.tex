\chapter{Week 1}
\setcounter{weekpage}{1}
\thispagestyle{plainweek}


\section{Introduction}


\begin{enumerate}


\item Notation and definitions:

\begin{itemize}

\item The \textbf{patial derivative} of $f$ with respect to $x$ is denoted
\[\frac{\partial f}{\partial x} = \frac{\partial}{\partial x}f = f_{x}\]

\item \textbf{Gradient} of $f(x,y,z)$
\[\nabla f (x,y,z) = (f_{x}(x,y,z),f_{y}(x,y,z),f_{z}(x,y,z)).\]

\item \textbf{Laplacian} of $f(x,y,z)$
\[\Delta f (x,y,z)= f_{xx}(x,y,z)+f_{yy}(x,y,z)+f_{zz}(x,y,z).\]

\item \textbf{Partial differential equation (PDE)} for unknown $u(x,y)$
\[F(x,y,u,u_{x},u_{y},u_{xx},u_{xy},u_{yy},u_{xxx},\cdots)=0.\]

\item \textbf{Linear differential operator} $L$ satisfies
\[L(u+v)=Lu+Lv \quad \text{and} \quad L(\lambda u) = \lambda u.\]

\item \textbf{Linear PDE} for unknown $u$
\[Lu=f,\]
where $L$ is a linear differential operator and function $f$ does not depend on $u$ or any of its derivatives. The equation is \textbf{homogeneous} if $f=0$, and \textbf{nonhomogeneous} if $f\neq0$.

\item The \textbf{order of a PDE} is the highest order derivative in the equation.
\end{itemize}


\newpage 
\item \textit{Example 1.1.}

Find the dimension and order of the following PDEs. Which are linear, and which are homogeneous?

\begin{itemize}
    \item {Heat equation:} 
    \[u_{t} = D u_{xx} + f(x)\]
    \vspace{20pt}

    \item {Wave equation:}
    \[u_{tt} - c u_{xx} =0 \]
    \vspace{20pt}

    \item {Laplace equation:}
    \[u_{xx} + u_{yy} = 0 \]
    \vspace{20pt}

    \item {Advection equation:}
    \[ \frac{\partial u}{\partial x}+ \frac{\partial u}{\partial y} = 0\]
    \vspace{20pt}

    \[ \frac{\partial^{2} u}{\partial x^{2}}+ e^{y}\sin(z)\frac{\partial^{2} u}{\partial x \partial z} = u\]
    \vspace{40pt}

    \[ \frac{\partial^{2} u}{\partial x \partial y} = \sin(u)\]
    \vspace{20pt}
    \item {KdV equation:} \[u_{t} + uu_{xx} + u_{xxx} =1\]
    \vspace{20pt}
\end{itemize}


\newpage

\item The \textbf{second-order linear constant-coefficiens homogeneous PDEs}
\[au_{xx}+2bu_{xy}+cu_{yy}+du_{x}+eu_{y}+fu=0\]
is said to be
\begin{itemize}
    \item \textbf{elliptic} iff $ac-b^{2}>0$.
    \item \textbf{parabolic} iff $ac-b^{2}=0$.
    \item \textbf{hyperbolic} iff $ac-b^{2}<0$.
\end{itemize}

\item \textit{Example 1.2.} 

Classify the following second-order linear PDEs.
\begin{itemize}
    \item $u _{t} + 2u _{tt} + 3u _{xx} = 0$
    \vspace{100pt}

    \item $17u_{ yy} + 3u _{x} + u = 0$
    \vspace{100pt}

    \item $4u _{xy} + 2u _{xx} + u _{yy} = 0$
    \vspace{100pt}

    \item $u _{yy} - u _{xx} - 2u _{xy} = 0$
    \vspace{100pt}

\end{itemize}



\newpage

\item \textbf{Superposition principle}.
If $u_{1}$ and $u_{2}$ are solutions to $Lu=0$, so is $c_{1}u_{1}+c_{2}u_{2}$.

\item \textbf{Theorem 1.2} If $u_{p}$ is a particular solution to $Lu=f$ and $u_{h}$ is the solutions to $L=u$, then $u=cu_{h}+u_{p}$ is a solution $Lu=f$ for any $c$.

\item \textit{Example 1.7. (Burger’s Equation)}

Consider the following two-dimensional first-order nonlinear PDE:
\[u_{x}+uu_{y}=0\]
and solutions
\[u_{1}(x,y)=1 \quad \text{and} \quad u_{2}(x,y)=\frac{y}{1+x}.\]


\vspace{200pt}

Consider the nonhomogeneous case:
\[u_{x}+uu_{y}=\frac{y^{2}-1}{x^{2}y^{3}}\]
with particular solution
\[u_{p}(x,y)=-\frac{1}{xy}.\]





\newpage

\item \textbf{Conditions}: a PDE can have
\begin{itemize}
    \item \textbf{Initial conditions}: value at time $t=0$, i.e., $u(x,y,0)=u_{0}(x,y)$.
    \item \textbf{Boundary conditions}: value on the boundary $\partial\Omega$ for all time
    \begin{itemize}
        \item Dirichlet: $u=g$ on $\partial\Omega$. Homogeneous if $g=0$.
        \item Neumann: $\frac{\partial u}{\partial n} =g$ on $\partial\Omega$. Homogeneous if $g=0$.
        \item Robin: $\alpha u +\beta \frac{\partial u}{\partial n} = g$ on $\partial\Omega$. Homogeneous if $g=0$.
    \end{itemize}

\end{itemize}


\item A \textbf{Boundary Value Problem} BVP is a PDE with boundary conditions.


\item A \textbf{steady-state solution} to a BVP does not depend on time, i.e., $u(x,t)=\tilde u (x)$.

\item \textit{Example 1.10.}

Find the steady-state solution to the following PDE on $[0, 2\pi]$ :
\begin{align*}
    & u_{ t} = 3u _{xx} + 9 \sin x, \\
    & u(x, 0) = 9 \sin x, \\
    & u(0, t) = 9, \\
    & u _{x} (2\pi, t) = 0.
\end{align*}

\newpage

\item \textbf{Exercise 15.1}

Show that the function
\[u = \frac{1}{\sqrt{x^{2}+y^{2}+z^{2}}}\]
is harmonic; that is, it is a solution to the three-dimensional Laplace equation $\Delta u = 0$.

\newpage

\item \textbf{Exercise 15.4}

Compute the Laplacian of the function
\[u(x,y)=\log\left( x^{2} +y^{2}\right)\]
in an appropriate coordinate system and decide if the given function satisfies Laplace's equation $\nabla ^{2} u = 0$.


\end{enumerate}




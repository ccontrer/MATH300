\chapter{Week 13}
\setcounter{weekpage}{1}
\thispagestyle{plainweek}

\section{Summary}

\begin{enumerate}


\item \textbf{Separations of variables}
\begin{enumerate}[1)]
\item Write $u(x, t) = X(x)T (t)$.
\item Solve the Sturm-Liouville problem for $X(x)$.
\item Solve the corresponding time problem for $T (t)$.
\item Use superposition.
\item Use the initial conditions.
\end{enumerate}

\item \textbf{Definition 2.10}.  The \textbf{Fourier series}  of $f$ on $(a, b)$ is given by
\[f(x) \sim a_{0}+\sum_{n=1}^{\infty}a_{n}\cos \frac{n\pi x}{l} + b_{n}\sin \frac{n\pi x}{l},\]
where $l=(b-a)/2$ and
\[a_{0} = \frac{1}{2l}\int_{a}^{b}f(x)dx, \quad a_{n} = \frac{1}{l}\int_{a}^{b}f(x)\cos\frac{n\pi x}{l}dx,\quad b_{n} = \frac{1}{l}\int_{a}^{b}f(x)\sin\frac{n\pi x}{l}dx,\quad n\geq 1,\]
are called the \textbf{Fourier coefficients} of $f$.

% \item Let function $f$ defined on $(0,l)$.
% \begin{itemize}
%     \item[(i)] The \textbf{Fourier sine series} for $f$ is
%     \[f(x) \sim \sum_{n=1}^{\infty}b_{n}\sin\frac{n\pi x}{l},\]
%     where
%     \[b_{n} = \frac{2}{l}\int_{0}^{l}f(x)\sin\frac{n\pi x}{l}dx \quad \text{for} \quad n \geq 1. \]
%     Note that this defines $f_{\text{even}}$, the  odd extension of $f$ on $(-l,l)$.
%     \item[(ii)] The \textbf{Fourier cosine series} for $f$ is
%     \[f(x) \sim a_{0} + \sum_{n=1}^{\infty}a_{n}\sin\frac{n\pi x}{l},\]
%     where
%     \[a_{0} = \frac{1}{l}\int_{0}^{l}f(x)dx, \quad \text{and} \quad a_{n} = \frac{2}{l}\int_{0}^{l}f(x)\cos\frac{n\pi x}{l}dx\quad \text{for} \quad n \geq 1. \]
%     Note that this defines $f_{\text{even}}$, the even extension of $f$ on $(-l,l)$.
% \end{itemize}



% \item Standard homogeneous Heat and Wave equations
% 
% \end{enumerate}
% 
% 
% \begin{minipage}{\linewidth}
%     \begin{minipage}{0.5\linewidth}
%         \textbf{Heat eq. with Dirichlet BCs}
%         \[
%         \begin{aligned}
%             & u_{t} = k  u_{xx}, \quad 0<x<a , \quad t>0, \\
%             & u(0,t)=0, \quad t>0,\\
%             & u(a,t)=0, \quad t>0, \\
%             & u(x,0)=f(x), \quad 0<x<a.
%         \end{aligned}
%         \]
%         The solution has the form
%         \[u(x,t)=\sum_{n=1}^{\infty} b_{n}e^{-\left(\frac{n\pi}{a}\right)^{2}kt}\sin \frac{n\pi x}{a}.\]
%     \end{minipage}
%     \begin{minipage}{0.5\linewidth}
%         \textbf{Heat equation with Neumann BCs}
%         \[
%         \begin{aligned}
%             & u_{t} = k  u_{xx}, \quad 0<x<a , \quad t>0, \\
%             & u_{x}(0,t)=0, \quad t>0,\\
%             & u_{x}(a,t)=0, \quad t>0, \\
%             & u(x,0)=f(x), \quad 0<x<a.
%         \end{aligned}
%         \]
%         The solution has the form
%         \[u(x,t)=a_{0}+\sum_{n=1}^{\infty} a_{n}e^{-\left(\frac{n\pi}{a}\right)^{2}kt}\cos \frac{n\pi x}{a}.\]
%     \end{minipage} \\
%     
%     \vspace{20pt}
%     \begin{minipage}{0.5\linewidth}
%         \textbf{Wave equation with Dirichlet BCs}
%         \[
%         \begin{aligned}
%             & u_{tt} = c^{2}  u_{xx}, \quad 0<x<a , \quad t>0, \\
%             & u(0,t)=0, \quad t>0,\\
%             & u(a,t)=0, \quad t>0, \\
%             & u(x,0)=f(x), \quad 0<x<a, \\
%             & u_{t}(x,0)=g(x), \quad 0<x<a.
%         \end{aligned}
%         \]
%         The solution has the form
%         \begin{multline*}
%             u(x,t)=\\ 
%             \sum_{n=1}^{\infty} \left(a_{n}\cos \frac{n\pi c t}{a} + b_{n}\sin \frac{n\pi c t}{a} \right) \sin \frac{n\pi x}{a}.
%         \end{multline*}
%     \end{minipage}
%     \begin{minipage}{0.5\linewidth}
%         \textbf{Wave equation with Neumann BCs}
%         \[
%         \begin{aligned}
%             & u_{tt} = c^{2}  u_{xx}, \quad 0<x<a , \quad t>0, \\
%             & u_{x}(0,t)=0, \quad t>0,\\
%             & u_{x}(a,t)=0, \quad t>0, \\
%             & u(x,0)=f(x), \quad 0<x<a, \\
%             & u_{t}(x,0)=g(x), \quad 0<x<a.
%         \end{aligned}
%         \]
%         The solution has the form
%         \begin{multline*}
%             u(x,t)= a_{0} + \\ 
%             \sum_{n=1}^{\infty} \left(a_{n}\cos \frac{n\pi c t}{a} + b_{n}\sin \frac{n\pi c t}{a} \right) \cos \frac{n\pi x}{a}.
%         \end{multline*}
%     \end{minipage}
% \end{minipage}
% 
% 
% \begin{enumerate}
% 
% \setcounter{enumi}{3}

\item \textbf{Method of Characteristics}

Consider the first-order linear time-dependent problem of the form
\[
\begin{aligned}
    & \frac{\partial u}{\partial t} + B(x,t)\frac{\partial u}{\partial x} = C(x,t,u) , \quad -\infty<x<\infty, \quad t>0, \\
    & u(x,0)=f(x).
\end{aligned}
\]
The method of characteristic consists on solving the \textbf{characteristic equations}
\[
\begin{aligned}
    & \frac{d x}{d t} = B(x,t),\\
    & \frac{d u}{d t} = C(x,t,u),
\end{aligned}
\]
and then using the initial condition.


\item Consider the one-dimensional wave equation
\[\frac{\partial^{2} u}{\partial t^{2}} = c^{2} \frac{\partial^{2} u}{\partial x^{2}}, \quad -\infty<x<\infty, \quad t>0, \quad u(x,0)=f(x), \quad \frac{\partial u}{\partial t}(x,0)=g(x).\]
\textbf{d'Alembert's solution} is given by
\[u(x,t)= \frac{1}{2}[f(x+ct)+f(x-ct)] + \frac{1}{2c} \int _{x-ct}^{x+ct} g(\mu)\, d\mu.\]

\item Consider the one-dimensional wave equation
\[
\begin{aligned}
& \frac{\partial^{2} u}{\partial t^{2}} = c^{2} \frac{\partial^{2} u}{\partial x^{2}}, \quad 0<x<l, \quad t>0, \\
& u(0,t)=0, \quad  u(l,t)=0, \quad u(x,0)=f(x), \quad \frac{\partial u}{\partial t}(x,0)=g(x).\end{aligned}
\]
\textbf{d'Alembert's solution} is given by
\[u(x,t)= \frac{1}{2}[\bar f_{\text{odd}}(x+ct)+ \bar f_{\text{odd}}(x-ct)] + \frac{1}{2c}\int _{x-ct}^{x+ct} \bar g_{\text{odd}}(\mu)\, d\mu,\]
where $\bar f_{\text{odd}}$ and $\bar g _{\text{odd}}$ are the $2l$-periodic extension of $f$ and $g$, respectively.



\item A regular \textbf{Sturm-Liouville problem},
\begin{align*}
& (p(x)\phi ' ) ' + [q(x) + \lambda \sigma(x)]\phi = 0, \quad a < x < b,\\
& \alpha _1 \phi(a) + \beta _1 \phi ' (a) = 0, \\
& \alpha _2 \phi(b) + \beta _2 \phi ' (b) = 0,
\end{align*}
has eigenvalues $\lambda_{n}$ and corresponding eigenfunctions $\phi_{n}$ for $n\geq 1$.

\item \textbf{Theorem 4.6.} \textit{(Dirichlet's Theorem)}

If $f$ is piecewise smooth on $[a, b]$, the \textbf{generalized Fourier series},
\[
f(x)\sim \sum_{n=1}^{\infty} c_{n}\phi_{n}(x), \quad 
\text{where}, \quad
c_{n}=\frac{\langle f, \phi_{n}\rangle}{\|\phi_{n}\|^{2}} = \frac{1}{\|\phi_{n}\|^{2}}\int_{a}^{b} f(x)\phi_{n}(x)\sigma(x)\,dx,
\]
for $n \geq 1$, converges pointwise to $[f (x ^{+} ) + f (x ^{-} )]/2$ for each $x \in (a, b)$.



\item \textbf{Method of eigenfunction expansions} 
\begin{enumerate}[1)]
    \item Identify the eigenfunctions $\phi_{n}$ associated to the problem.
    \item Assume a solution of the form $u(x,t) = \sum_{n=1}^{\infty} a_{n}(t)\phi_{n}(x)$. 
    \item Expand initial conditions and other related functions using generalized Fourier series.
    \item Substitute into the equation to solve for $a_{n}(t)$.
\end{enumerate}


\item \textbf{Theorem 4.7.} 

$\lambda _{n}$ can be calculated from the \textbf{Rayleigh quotient}:
\[
\lambda_{n} = \frac{ \displaystyle -p(x)\phi_{n}(x)\phi'_{n}(x)\Big|_{a}^{b} + \int_{a}^{b}(p(x)\phi'_{n}(x)^{2} - q(x)\phi_{n}(x)^{2}) \, dx }{ \displaystyle \int_{a}^{b}\phi_{n}(x)^{2}\sigma(x)\, dx}.
\]

\newpage
\item Summary of Sturm-Liouville problems.

\end{enumerate}

\begin{center}\small
\begin{tabular}{|c|c|c|c|}
    \hline
    \parbox[c][30pt]{90pt}{\centering \textbf{Model Type}} & 
    \parbox[c][30pt]{90pt}{\centering \textbf{S-L Problem}} & \parbox[t]{90pt}{\centering \textbf{Spectrum}} & 
    \parbox[c][30pt]{90pt}{\centering \textbf{Eigenfunctions}} \\
    \hline
    
    \parbox[c][30pt]{90pt}{\centering \textbf{Homogeneous \\ \vspace{10pt} Dirichlet B.C.}} & 
    \parbox[c][60pt]{90pt}{\centering $\phi''(x) + \lambda \phi(x)=0$ \\ \vspace{10pt} $\phi(0)=\phi(l)=0$}
    & 
    \parbox[c][60pt]{90pt}{\centering $\displaystyle \lambda_{n}=\left( \frac{n\pi}{l} \right)^{2}$ \\ \vspace{10pt} $n=1, 2, \cdots$} & 
    \parbox[c][60pt]{90pt}{\centering $\displaystyle \phi_{n} = \sin\frac{n\pi x}{l}$ \\ \vspace{10pt} $n=1, 2, \cdots$} \\
    \hline 
    
    \parbox[c][30pt]{90pt}{\centering \textbf{Homogeneous \\ \vspace{10pt} Neumann B.C.}} & 
    \parbox[c][60pt]{90pt}{\centering $\phi''(x) + \lambda \phi(x)=0$ \\ \vspace{10pt} $\phi'(0)=\phi'(l)=0$}
    & 
    \parbox[c][60pt]{90pt}{\centering $\displaystyle \lambda_{n}=\left( \frac{n\pi}{l} \right)^{2}$ \\ \vspace{10pt} $n=0, 1, \cdots$} & 
    \parbox[c][60pt]{90pt}{\centering $\displaystyle \phi_{n} = \cos\frac{n\pi x}{l}$ \\ \vspace{10pt} $n=0, 1, \cdots$} \\
    \hline
    
    \parbox[c][30pt]{90pt}{\centering \textbf{Mixed \\ \vspace{10pt} Type I}} & 
    \parbox[c][60pt]{90pt}{\centering $\phi''(x) + \lambda \phi(x)=0$ \\ \vspace{10pt} $\phi(0)=\phi'(l)=0$}
    & 
    \parbox[c][60pt]{100pt}{\centering $\displaystyle \lambda_{n}=\left( \frac{(2n-1)\pi}{2l} \right)^{2}$ \\ \vspace{10pt} $n=1, 2, \cdots$} & 
    \parbox[c][60pt]{100pt}{\centering $\displaystyle \phi_{n} = \sin\frac{(2n-1)\pi x}{2l}$ \\ \vspace{10pt} $n=1, 2, \cdots$} \\
    \hline
    
    \parbox[c][30pt]{90pt}{\centering \textbf{Mixed \\ \vspace{10pt} Type II}} & 
    \parbox[c][60pt]{90pt}{\centering $\phi''(x) + \lambda \phi(x)=0$ \\ \vspace{10pt} $\phi'(0)=\phi(l)=0$}
    & 
    \parbox[c][60pt]{100pt}{\centering $\displaystyle \lambda_{n}=\left( \frac{(2n-1)\pi}{2l} \right)^{2}$ \\ \vspace{10pt} $n=1, 2, \cdots$} & 
    \parbox[c][60pt]{100pt}{\centering $\displaystyle \phi_{n} = \cos\frac{(2n-1)\pi x}{2l}$ \\ \vspace{10pt} $n=1, 2, \cdots$} \\
    \hline
    
    \parbox[c][30pt]{90pt}{\centering \textbf{Periodicity \\ \vspace{10pt} conditions}} & 
    \parbox[c][80pt]{90pt}{\centering $\phi''(\theta) + \lambda \phi(\theta)=0$ \\ \vspace{10pt} $\phi(-\pi)=\phi(\pi)$ \\ \vspace{10pt} $\phi'(-\pi)=\phi'(\pi)$}
    & 
    \parbox[c][60pt]{100pt}{\centering $\displaystyle \lambda_{n}=n^{2}$ \\ \vspace{10pt} $n=0, 1, \cdots$} & 
    \parbox[c][60pt]{120pt}{\centering $ \phi_{n} = a_{n}\cos n \theta + b_{n}\sin n \theta$ \\ \vspace{10pt} $n=0, 1, \cdots$} \\
    \hline
    
    \parbox[c][30pt]{90pt}{\centering \textbf{Bessel \\ \vspace{10pt} equation}} & 
    \parbox[c][80pt]{130pt}{\centering $r^{2}u'' + r u'+ (\lambda r^{2} - m^{2}) u=0$ \\ \vspace{10pt} $u(a)=0$ and $|u(r)|$ \\ \vspace{10pt} bounded as $r\to 0^{+}$}
    & 
    \parbox[c][60pt]{100pt}{\centering $\displaystyle \lambda_{mn}=\left( \frac{z_{mn}}{a} \right)^{2}$ \\ \vspace{10pt} $n=1, 2, \cdots$} & 
    \parbox[c][60pt]{120pt}{\centering $ u_{mn} = a_{mn}J_{m}\left( \frac{z_{mn}r}{a} \right)$ \\ \vspace{10pt} $n=1, 2, \cdots$} \\
    \hline
    
    \parbox[c][30pt]{90pt}{\centering \textbf{Legendre \\ \vspace{10pt} equation}} & 
    \parbox[c][80pt]{120pt}{\centering $(\sin\theta \,v')' + \lambda \sin \theta v=0$ \\ \vspace{10pt} $|v(\theta)|$ and $|v'(\theta)|$ \\ \vspace{10pt} bounded as $\theta \to \pm 1$}
    & 
    \parbox[c][60pt]{100pt}{\centering $\displaystyle \lambda_{n}=n(n+1)$ \\ \vspace{10pt} $n=0, 1, \cdots$} & 
    \parbox[c][60pt]{120pt}{\centering $ v_{n} = a_{n}P_{ n} (\cos \theta )$\\ \vspace{10pt} $n=0, 1, \cdots$} \\
    \hline
\end{tabular}
\end{center}


\begin{enumerate}

\setcounter{enumi}{9}

\item The \textbf{Laplacian in polar coordinates} is
\[\nabla^{2} u = \frac{1}{r}\frac{\partial}{\partial r}\left( r\frac{\partial u}{\partial r} \right) +\frac{1}{r^{2}}\frac{\partial^{2} u}{\partial \theta^{2}} = \frac{\partial^{2} u}{\partial r^{2}} + \frac{1}{r}\frac{\partial u}{\partial r} + \frac{1}{r^{2}}\frac{\partial^{2} u}{\partial \theta^{2}}.\]

\item The \textbf{Laplacian in spherical coordinates} is
\[
\begin{aligned}\nabla^{2} u & = \frac{1}{r^{2}}\frac{\partial}{\partial r}\left( r^{2} \frac{\partial u}{\partial r} \right) + \frac{1}{r^{2}\sin \theta} \frac{\partial}{\partial \theta}\left( \sin\theta \frac{\partial u}{\partial \theta} \right) + \frac{1}{r^{2}\sin^{2}\theta} \frac{\partial^{2} u}{\partial \phi^{2}} \\
& = \frac{\partial^{2} u}{\partial r^{2}} + \frac{2}{r}\frac{\partial u}{\partial r} + \frac{1}{r^{2}}\left( \frac{\partial^{2} u}{\partial \theta^{2}} + \cot \theta \frac{\partial u}{\partial \theta} + \csc^{2}\theta \frac{\partial^{2}u}{\partial \phi^{2}} \right).
\end{aligned}
\]


\item \textbf{Fourier Integral Representation} of $f$ on $(-\infty,\infty)$

\[f(x) = \int _{0}^{\infty} \left[A(\omega)\cos \omega x + B(\omega) \sin \omega x \right]\, d\omega\]
where 
\[A(\omega) = \frac{1}{\pi}\int _{-\infty}^{\infty} f(x)\cos \omega x \, dx ,\quad B(\omega) = \frac{1}{\pi} \int_{-\infty}^{\infty} f(x) \sin \omega x \,dx.\]

\item \textbf{Fourier Cosine Integral Representation} of $f$ on $[0,\infty)$
\[f(x) = \int _{0}^{\infty} A(\omega)\cos \omega x \, d\omega, \quad \text{where} \quad A(\omega) = \frac{2}{\pi}\int _{0}^{\infty} f(x)\cos \omega x \, dx .\]


\item \textbf{Fourier Sine Integral Representation} of $f$ on $[0,\infty)$
\[f(x) = \int _{0}^{\infty} B(\omega)\sin \omega x \, d\omega, \quad \text{where} \quad B(\omega) = \frac{2}{\pi}\int _{0}^{\infty} f(x)\sin \omega x \, dx .\]


\item \textbf{Fourier transform} of $f(x)$ on $(-\infty,\infty)$,
\[\widehat{f}(\omega) = \mathcal{F} \, [f(x)] \,(\omega) =  \frac{1}{2\pi} \int_{-\infty} ^{\infty}f(x) \,e^{i\omega x} \,dx, \quad -\infty < \omega <\infty.\]
\[{f}(x) = \mathcal{F}^{-1} \, [\widehat{f}(\omega)] \,(x) =  \int_{-\infty} ^{\infty}\widehat{f}(\omega) \,e^{-i\omega x} \,d\omega, \quad -\infty < x <\infty.\]

\item \textbf{Fourier cosine transform} of $f (x)$ on $[0,\infty)$
\[\mathcal{C}\,[f(x)](\omega)=\frac{2}{\pi} \int_{0}^{\infty}f(x)\cos\omega x \, dx, \quad \mathcal{C}^{-1}\,[g(\omega)](x)= \int_{0}^{\infty}g(\omega)\cos\omega x \, d\omega.\]

\item \textbf{Fourier sine transform} of $f (x)$ on $[0,\infty)$
\[\mathcal{S}\,[f(x)](\omega)=\frac{2}{\pi} \int_{0}^{\infty}f(x)\sin\omega x \, dx, \quad \mathcal{S}^{-1}\,[g(\omega)](x)= \int_{0}^{\infty}g(\omega)\sin\omega x \, d\omega.\]

\item \textbf{Gauss kernel}
\[g(x,t) = \frac{1}{\sqrt{4\pi k t}}e^{-\frac{x^{2}}{4 k t}}.\]

\item \textbf{Error function}
\[\text{erf}(x):= \frac{2}{\sqrt{\pi}}\int_{0}^{x}e^{-t^{2}}\,dt.\]

\end{enumerate}

\newpage

\section{Final review}

\begin{enumerate}

\item \textit{Example 3.4 modified.} Consider the following nonhomogeneous one-dimensional heat equation:

\[
\begin{aligned}
    & \frac{\partial u}{\partial t}  = k \frac{\partial^{2} u}{\partial x^{2}} + f(x,t) , \quad 0 < x < 2, \quad t>0, \\
    & u(0, t) = 0, \\
    & \frac{\partial u}{\partial x}(2, t) = 0, \\
    & u (x, 0) = g(x),
\end{aligned}
\]
where $f(x, t)$ is a continuous function of $x$ and $t$.

\newpage

\textit{(continue Example 3.4.)}

\newpage

\item \textit{Example 3.4 modified part 2.} Consider the following nonhomogeneous one-dimensional heat equation:

\[
\begin{aligned}
    & \frac{\partial u}{\partial t}  = k \frac{\partial^{2} u}{\partial x^{2}} + f(x,t) , \quad 0 < x < 2, \quad t>0, \\
    & u(0, t) = a(t), \\
    & \frac{\partial u}{\partial x}(2, t) = b(t), \\
    & u (x, 0) = g(x)
\end{aligned}
\]
where $f(x, t)$ is a continuous function of $x$ and $t$, and $a(t)$ and $b(t)$ are
continuously differentiable functions of $t$.


\newpage

\item \textbf{Exercise 19.9.} Consider torsional oscillations of a homogeneous cylindrical shaft. If $\omega(x, t)$ is the angular displacement at time $t$ of the cross section at $x$, then
\[\frac{\partial^{2} \omega}{\partial t^{2}} = a^{2} \frac{\partial^{2} \omega}{\partial x^{2}}, \quad 0<x<l,\quad t>0, \]
where the initial conditions are
\[\omega(x,0)=f(x) \quad \text{and} \quad \frac{\partial \omega}{\partial t} (x,0)=0,\]
and the ends of the shaft are fixed elastically:

\[\frac{\partial \omega}{\partial x} (0,t) -\alpha\, \omega(0,t)=0 \quad \text{and} \quad \frac{\partial \omega}{\partial x} (l,t) + \alpha\, \omega(l,t)=0 \]
with $\alpha$ a positive constant.

\begin{enumerate}[(a)]
\item Why is it possible to use separation of variables to solve this problem?

\item Use separation of variables and show that one of the resulting problems is a regular Sturm-Liouville problem.

\item Show that all of the eigenvalues of this regular Sturm-Liouville problem are positive.

\end{enumerate}

\textbf{Note:} You do not need to solve the initial value problem, just answer the questions (a), (b), and
(c).

\newpage

\textit{(continue Exercise 19.19.)}

\newpage


\item How to solve PDE's in practice?

\end{enumerate}






\chapter{Week 3}
\setcounter{weekpage}{1}
\thispagestyle{plainweek}

\section{Fourier Series}

\begin{enumerate}




\item \textbf{Theorem 2.3}. \textit{(Dirichlet's Theorem)}

Let $f (x)$ be piecewise smooth on the interval $(−l, l)$. The Fourier series 
\[a_{0}+\sum_{n=1}^{\infty}a_{n}\cos \frac{n\pi x}{l} + b_{n}\sin \frac{n\pi x}{l},\]
where
\[a_{0} = \frac{1}{2l}\int_{-l}^{l}f(x)dx, \quad a_{n} = \frac{1}{l}\int_{-l}^{l}f(x)\cos\frac{n\pi x}{l}dx,\quad b_{n} = \frac{1}{l}\int_{-l}^{l}f(x)\sin\frac{n\pi x}{l}dx,\quad n\geq 1,\]
has the following properties:
\begin{itemize}

\item[(i)] If $f (x)$ is continuous at $x _{0}$ , where $-l < x _{0} < l$, then
\[f(x_{0})=a_{0}+\sum_{n=1}^{\infty}a_{n}\cos \frac{n\pi x_{0}}{l} + b_{n}\sin \frac{n\pi x_{0}}{l};\]
that is, the Fourier series converges to $f (x _{0} )$.
\item[(ii)] If $f (x)$ has a jump discontinuity at $x _{0}$ , where $−l < x 0 < l$, then
\[\frac{f(x_{0}^{+})+f(x_{0}^{-})}{2} = a_{0}+\sum_{n=1}^{\infty}a_{n}\cos \frac{n\pi x_{0}}{l} + b_{n}\sin \frac{n\pi x_{0}}{l};\]
that is, the Fourier series converges to the \textbf{average} or \textbf{mean} of the jump.
\item[(iii)] At the endpoints $x 0 = \pm l$, the Fourier series converges to
\[\frac{f(-l^{+})+f(l^{-})}{2}.\]
\end{itemize}
As usual, we write
\[f(x)\sim a_{0}+\sum_{n=1}^{\infty}a_{n}\cos \frac{n\pi x}{l} + b_{n}\sin \frac{n\pi x}{l},\]
and say that $f (x)$ is \textbf{represented by its Fourier series} on the interval $(−l, l)$.



The Fourier series defines a $2l$-periodic extension of $f(x)$ for all $x\in \mathbb{R}$.

\newpage


\item \textbf{Exercise 11.5} 

Compute the Fourier series of the $2\pi$-periodic function $f$ given by
\[
f(x)=
\begin{cases}
1, & 0<x<\pi/2, \\
0, & \pi/2<|x|< \pi, \\
-1, & -\pi/2<x<0.\\
\end{cases}
\]
For which values of $x$ does the Fourier series converge to $f$? Sketch the graph of the Fourier.





\newpage

\item \textbf{Exercise 11.6} 

Compute the Fourier series of the $2\pi$-periodic function $f$ given by $f(x)=|\cos(x)|$. For which values of $x$ does the Fourier series converge to $f$? Sketch the graph of the Fourier.


\newpage 

(continue)


\newpage 


\item \textbf{Exercise 11.7} 

Consider the parabola $f(x)=x^{2}$ on $[-a,a]$ and show that the Fourier series of $f$ is given by 
\[\frac{a^{2}}{3} - \frac{4a^{2}}{\pi^{2}} \left[ \cos\left( \frac{\pi x}{a} \right) - \frac{1}{2^{2}}\cos\left( \frac{2 \pi x}{a} \right) + \frac{1}{3^{2}}\cos\left( \frac{3 \pi x}{a} \right) +\cdots \right]. \]
Find its values and the points of discontinuity.



\newpage



\item \textbf{Theorem 2.4.} \textit{(Uniqueness of Fourier Series)} 

If $f$ is $2 l$-periodic and piecewise smooth on the interval $(-l,l)$, its Fourier series is unique.



\item \textbf{Theorem 2.5.} \textit{(Linearity of Fourier Series)}

If $f$ and $g$ are piecewise continuous on $(-l,l)$ and $c _{1}$ and $c _{2}$ are scalars, the Fourier series of
\[c_{1}f+c_{2}g\]
is the sum of $c _{1}$ times the Fourier series of $f (x)$ and $c _{2}$ times the Fourier series of $g(x)$.



\item \textbf{Theorem 2.8.} \textit{(Term-by-Term Differentiation of Fourier Series)}

Let $f$ be a function such that
\begin{itemize}
\item[(i)] $f$ is continuous on the interval $-\pi \leq x \leq \pi$;
\item[(ii)] $f (-\pi) = f (\pi)$; and
\item[(iii)] $f'$ is piecewise smooth on the interval $-\pi < x < \pi$.
\end{itemize}

% The Fourier series representation
% \[f(x) = a_{0}+\sum_{n=1}^{\infty}a_{n}\cos nx + b_{n}\sin nx,\]
% is differentiable at each point $x _{0}$ with $-\pi < x _{0} < \pi$ at which $f''(x _{0} )$
% exists, and
% \[f'(x_{0}) = \sum_{n=1}^{\infty}n(-a_{n}\sin nx_{0} + b_{n}\cos nx_{0}).\]
% At a point $x _{0}$ with $-\pi < x _{0} < \pi$ at which $f '' (x _{0} )$ does not exist but where $f'$ has one-sided derivatives, the series above converges to 
% \[\frac{f'(x_{0}^{+})+f'(x_{0}^{-})}{2}.\]
The derivative of the Fourier series representation of $f$ is represented by
\[
f'(x) \sim
\begin{cases}
\displaystyle \sum_{n=1}^{\infty}n(-a_{n}\sin nx_{0} + b_{n} \cos nx_{0}), & \text{if  $f''(x _{0} )$
exists} \\
\displaystyle \frac{f'(x_{0}^{+})+f'(x_{0}^{-})}{2}, & \text{if $f '' (x _{0} )$ DNE but one-sided derivatives exist.}
\end{cases}
\]


\item \textbf{Theorem 2.9.} \textit{(Term-by-Term Differentiation of Fourier Cosine Series)}

Let $f$ be a function such that
\begin{itemize}
 \item[(i)] $f$ is continuous on the interval $0 \leq x \leq \pi$;
 \item[(ii)] $f'$ is piecewise continuous on the interval $0 < x < \pi$.
\end{itemize}
% The Fourier cosine series representation
% \[f(x) = a_{0}+\sum_{n=1}^{\infty}a_{n}\cos nx,\]
% is differentiable at each point $x _{0}$ with $0 < x _{0} < \pi$ at which $f''(x _{0} )$
% exists, and
% \[f'(x_{0}) = -\sum_{n=1}^{\infty}na_{n}\sin nx_{0}.\]
% At a point $x _{0}$ with $-\pi < x _{0} < \pi$ at which $f '' (x _{0} )$ does not exist but where $f'$ has one-sided derivatives, the series above converges to 
% \[\frac{f'(x_{0}^{+})+f'(x_{0}^{-})}{2}.\]
The derivative of the Fourier Cosine series representation of $f$ is represented by
\[
f'(x) \sim
\begin{cases}
\displaystyle -\sum_{n=1}^{\infty}na_{n}\sin nx_{0} , & \text{if  $f''(x _{0} )$
exists} \\
\displaystyle \frac{f'(x_{0}^{+})+f'(x_{0}^{-})}{2}, & \text{if $f '' (x _{0} )$ DNE but one-sided derivatives exist.}
\end{cases}
\]

\item \textbf{Theorem 2.10.} \textit{(Term-by-Term Differentiation of Fourier Sine Series)}

Let $f$ be a function such that
\begin{itemize}
\item[(i)] $f$ is continuous on the interval $0 \leq x \leq \pi$;
\item[(ii)] $f (0) = f (\pi)$; and
\item[(iii)] $f'$ is piecewise smooth on the interval $0 < x < \pi$.
\end{itemize}

% The Fourier sine series representation
% \[f(x) = \sum_{n=1}^{\infty} b_{n}\sin nx,\]
% is differentiable at each point $x _{0}$ with $0 < x _{0} < \pi$ at which $f''(x _{0} )$
% exists, and
% \[f'(x_{0}) = \sum_{n=1}^{\infty}n(-a_{n}\sin nx_{0} + b_{n}\cos nx_{0}).\]
% At a point $x _{0}$ with $-\pi < x _{0} < \pi$ at which $f '' (x _{0} )$ does not exist but where $f'$ has one-sided derivatives, the series above converges to 
% \[\frac{f'(x_{0}^{+})+f'(x_{0}^{-})}{2}.\]
The derivative of the Fourier Sine series representation of $f$ is represented by
\[
f'(x) \sim
\begin{cases}
\displaystyle \sum_{n=1}^{\infty}nb_{n}\cos nx_{0} , & \text{if  $f''(x _{0} )$
exists} \\
\displaystyle \frac{f'(x_{0}^{+})+f'(x_{0}^{-})}{2}, & \text{if $f '' (x _{0} )$ DNE but one-sided derivatives exist.}
\end{cases}
\]



\item \textbf{Theorem 2.11.} \textit{(Term-by-Term Integration of Fourier Series)}

Let $f$ be piecewise continuous on the interval $-\pi < x < \pi$, and suppose that on $(-\pi,\pi)$
\[f(x) \sim a_{0}+\sum_{n=1}^{\infty}a_{n}\cos nx + b_{n}\sin nx,\]
then for $-\pi\leq x \leq \pi$
\[\int_{-\pi}^{\pi}f(t)dt = a_{0}(x+\pi) + \sum_{n=1}^{\infty}\frac{1}{n}\{a_{n}\sin nx - b_{n}[(-1)^{n+1} + \cos nx]\}.\]




\item \textbf{Exercise 11.8} 

Consider the $2a$-periodic function $f$ that is given on the interval $-a < x < a$ by $f (x) = x$. Show that the Fourier series of $f$ is given by

\[\frac{2a}{\pi}\sum_{n=1}^{\infty}\frac{(-1)^{n+1}}{n}\sin\left( \frac{n\pi x}{a} \right)\]
by differentiating the Fourier series in \textit{Exercise 11.7} term-by-term. Justify your work.



\newpage

\item \textbf{Euler's formula} in complex variables
\[e^{i\theta} = \cos \theta + i\sin \theta,\]
and complex trigonometric formulas
\[\cos \theta = \frac{e^{i\theta} + e^{-i\theta} }{2}= \cosh i\theta  \qquad \text{and} \qquad \sin \theta = \frac{e^{i\theta} - e^{-i\theta} }{2}=-i \sinh i\theta.\]



\item \textbf{Theorem 2.14.} The complex Fourier series for $f \in P W C(-l, l)$ is 

\[f(x) \sim \sum _{n=-\infty}^{\infty} c_{n}e^{i n\pi x/ l }, \qquad \text{where} \qquad c_{n}=\frac{1}{2l}\int_{-l}^{l}f(x)e^{-i n \pi x/l}dx, \quad n\in \mathbb{Z}.\]



\item \textit{Example 2.19.} Calculate the complex Fourier series for
\[f(x)=x, \quad -\pi<x<\pi,\]
and $f(x+2\pi)=f(x)$ otherwise.

\newpage

\null



\end{enumerate}




